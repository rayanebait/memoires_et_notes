\documentclass[12pt]{article}
\usepackage{amsmath, latexsym, amsfonts,amssymb, amsthm,
    amscd, geometry, xspace, enumerate,mathtools}
\title{Le théorème de densité de Chebotarev}
\date{Ann\'ee 2021-2022}
\author{Bait Rayane}

\setlength{\oddsidemargin}{-10mm}
\setlength{\evensidemargin}{5mm}
\setlength{\textwidth}{175mm}
\setlength{\headsep}{0mm}
\setlength{\topmargin}{0mm}
\setlength{\textheight}{220mm}
\setlength\parindent{24pt}

\newcommand{\Z}{\mathbb{Z}}
\newcommand{\R}{\mathbb{R}}
\newcommand{\rel}{\mathcal{R}}
\newcommand{\Q}{\mathbb{Q}}
\newcommand{\C}{\mathbb{C}}
\newcommand{\N}{\mathbb{N}}
\newcommand{{\K}}{\mathbb{K}}
\newcommand{\F}{\mathcal{F}}
\newcommand{\B}{\mathcal{B}}

\newcommand{\cL}{\mathcal{L}}
\newcommand{\D}{\mathcal{D}}

\theoremstyle{plain}
\newtheorem{thm}[subsubsection]{Th\'eor\`eme}
\newtheorem{lem}[subsubsection]{Lemme}
\newtheorem{prop}[subsubsection]{Proposition}
\newtheorem{propr}[subsubsection]{Propri\'et\'e}
\newtheorem{cor}[subsubsection]{Corollaire}
\newtheorem{intro}[section]{Introduction}
\newtheorem{thm2}[subsection]{Th\'eor\`eme}
\newtheorem{lem2}[subsection]{Lemme}
\newtheorem{prop2}[subsection]{Proposition}
\newtheorem{propr2}[subsection]{Propri\'et\'e}
\newtheorem{cor2}[subsection]{Corollaire}

\theoremstyle{definition}
\newtheorem{defn}[subsubsection]{D\'efinition}
\newtheorem{rmq}[subsubsection]{Remarque}
\newtheorem{conj}[subsubsection]{Conjecture}
\newtheorem{exmp}[subsubsection]{Exemples}
\newtheorem{quest}[subsubsection]{Exercices}
\newtheorem{defn2}[subsection]{D\'efinition}
\newtheorem{rmq2}[subsection]{Remarque}
\newtheorem{conj2}[subsection]{Conjecture}
\newtheorem{exmp2}[subsection]{Exemples}
\newtheorem{quest2}[subsection]{Exercices}

\theoremstyle{remark}
\newtheorem{rem}{Remarque}
\newtheorem{note}{Note}

\renewcommand*\contentsname{\begin{center}
        Table des matières
    \end{center}}

\begin{document}
\maketitle{}
\begin{center}
    Université de Paris\\
    Sous la direction de Herblot Mathilde et Galateau Aurélien
\end{center}
\newpage

\tableofcontents

\newpage
{\Large\bf 0 Introduction}\\
\newline
{\normalsize \indent Lors de l'étude de certains ensembles de nombres premiers
    l'une des premières questions posées est : En existe-t-il une infinité ? Une
    manière d'y répondre est d'en calculer la "proportion parmi tout les nombres
    premiers" en un sens à préciser. Intuitivement si cette proportion n'est pas
    nulle, notre ensemble est infini.\\ \indent Deux résultats utilisant cette
    méthode sont présentés ici, le premier est le théorème de la progression
    arithmétique de Dirichlet affirmant que dans toute progression arithmétique de
    la forme $a+bn$ avec $pgcd(a,b)=1$, une infinité de nombres premiers apparaît.
    Le second se présente comme suit : Soit $\K/\Q$ une extension finie galoisienne
    de groupe de Galois $G$. À tout nombre premier $p$ qui ne se ramifie pas dans
    $K$ correspond un morphisme $\sigma_p\in G$ défini à classe de conjugaison
    près. On pose la question suivante. Soit $\sigma\in G$, existe-t-il une
    infinité de nombres premiers $p$ tels que $\sigma\in C(\sigma_p)$, où
    $C(\sigma_p)$ est la classe de conjugaison de $\sigma_p$ dans $G$ ? C'est à
    cette question que le théorème de Chebotarev répond. \\ \indent Ce second
    théorème est rencontré relativement tard en théorie algébrique des nombres,
    beaucoup de résultats intermédiaires sont nécessaires à la compréhension des
    outils utilisés dans sa preuve. On en fera donc ici qu'un rappel partiel et on
    se référera au traitement de [2] des résultats classiques de théorie algébrique
    des nombres (chapitre 1,2,3,4), le chapitre 2 consistera donc en des rappels de
    ces chapitres. On admet en plus les résultats et définitions classiques de
    théorie de Galois.
    \indent  }
\newpage
\section{Le théorème de la progression arithmétique de Dirichlet}
\indent Dans cette partie on admet des résultats bien connus sur les fonctions
holomorphes ou les suites et séries de fonctions d'une variable complexe ainsi
que sur la théorie des caractères. On suivra ici [1], chapitre 6. \\
\indent\\\indent Le but de cette partie est de montrer le résultat suivant
:\begin{thm}
    Soit $a$ et $m$ deux entiers premiers entre eux. Alors il existe une infinité
    de nombre premier $p$ tel que $$p\equiv a\mod m$$
\end{thm}

\noindent Qui peut être précisé comme suit avec la notion de densité :
\begin{defn}
    Soit $P$ l'ensemble des nombres premiers de $\N$ et $A$ un sous-ensemble de
    $P$. On dira que $A$ a pour densité $d(A)$ lorsque la limite
    $$\lim_{s\rightarrow1,~s>1}(\sum_{p\in A}\frac{1}{p^s})/(\ln\frac{1}{s-1})$$
    existe et que cette limite vaut $d(A)$.
\end{defn}

\noindent Avec cette définition on voit alors que tout $A\subset P$ ayant une
densité non nulle est infini, en effet si $A$ était fini la quantité
$\sum_{p\in A}\frac{1}{p^s}$ serait finie et aurait alors pour densité $0$.\\

\begin{thm}
    Soit $a$ et $m$ deux entiers premiers entre eux et $P_a$ l'ensemble des nombres
    premiers $p$ tels que $p\equiv a\mod m$, alors $P_a$ a une densité et sa
    densité vaut $\frac{1}{\phi(m)}$, où $\phi(m)$ désigne la fonction indicatrice
    d'Euler.
\end{thm}

\indent \newline \noindent La première sous partie constituera une base pour
l'étude des fonctions $L$ et justifiera l'utilisation de cette définition pour
la densité. La deuxième sous-partie permettra, à l'aide des propriétés des
fonctions $L$, de donner un équivalent de $\sum_{p\in P_a}\frac{1}{p^s}$ pour
$s\rightarrow 1$ et ainsi prouver le théorème $1.0.3.$\noindent Commencons par
introduire quelques lemmes qui nous serons utiles par la suite.\begin{lem}
    (Lemme d'Abel) Soient $(a_n)$ et $(b_n)$ deux suites. Posons :
    $$A_{m,p}=\sum_{n=m}^{p}a_n~et~S_{m,m'}=\sum_{n=m}^{m'}a_nb_n$$
    On a alors : $$S_{m,m'}=\sum_{n=m}^{m'-1}A_{m,n}(b_n-b_{n+1})+A_{m,m'}b_{m'}$$
\end{lem}
{\bf \noindent Preuve :} La preuve consiste à remplacer $a_n$ par
$A_{m,n}-A_{m,n-1}$ puis à réorganiser les termes.

\begin{lem}
    Soient $0<\alpha<\beta$ et $z\in\C$ tel que \emph{Re}$(z)>0$. On a alors
    :$$\lvert e^{-\alpha z}-e^{-\beta
        z}\rvert\leq\lvert\frac{z}{\emph{Re}(z)}\rvert(e^{-\alpha
            \emph{Re}(z)}-e^{-\beta \emph{Re}(z)})$$
\end{lem}

{\bf\noindent  Preuve :} On a $$e^{-\alpha z}-e^{-\beta
    z}=z.\int_{\alpha}^{\beta}e^{-tz}dt$$ d'où en passant aux valeurs absolues
$$\lvert e^{-\alpha z}-e^{-\beta z}\rvert~\leq~\lvert
    z\rvert\int_{\alpha}^{\beta}e^{-t\emph{Re}(z)}=\lvert\frac{z}{\emph{Re}(z)}\rvert(e^{-\alpha
        \emph{Re}(z)}-e^{-\beta \emph{Re}(z)})$$qui est le résultat voulu.\qed

\subsection{Séries de Dirichlet et fonction zêta}

\begin{defn} Soit $(\lambda_n)$ une suite strictement croissante de nombres
    réels qui tend vers $+\infty$. On appelle série de Dirichlet d'exposants
    $(\lambda_n)$ toute série de la forme :$$\sum a_ne^{-\lambda_ns}~
        ,a_n\in\C,~s\in\C$$
\end{defn}

\noindent Soit $f(s)$ une série de Dirichlet où l'on aura supposé les
$\lambda_n$ positifs. Les résultats que l'on montrera sur $f(s)$ seront
toujours valables sur une série de Dirichlet quelconque car on peut se ramener
à notre cas en supprimant un nombre fini de termes, on appellera à nouveau par
$f(s)$ sa somme là ou elle converge. Pour $\rho\in\R$ on appelle $D(\rho)$ le
demi plan ouvert $\{z\in\C\mid \emph{Re}(z)>\rho\}$ de $\C$.

\begin{prop}
    Si $f$ converge en $s_0$, elle converge uniformément sur $D(Re(s_0))$.
\end{prop}

{\bf \noindent	Preuve :} Quitte à remplacer $s$ par $s-s_0$ on peut supposer
que $s_0=0$. L'hypothèse veut alors dire que $\sum a_n$ est convergente. Il
suffit de prouver que $f(s)$ converge uniformément dès que $Re(s)>0$ et $\lvert
    Arg(s)\rvert\leq\alpha$ où $\alpha<\frac{\pi}{2}$. On peut reformuler la
deuxième hypothèse en $$\frac{\lvert
        z\rvert}{\emph{Re}(z)}=\frac{1}{\cos(Arg(s))}\leq k$$ où $k\geq1$. Soit
$\epsilon>0$. Comme $\sum a_n$ converge, il existe $N$ tel que $\lvert
    A_{m,m'}\rvert\leq\epsilon$ dès que $m,m'\geq N$, les notations étant celles du
lemme d'Abel. On applique celui-ci avec $b_n=e^{-\lambda_n s}$, on obtient :
$$S_{m,m'}=\sum_{n=m}^{m'-1}A_{m,n}(e^{-\lambda_n
            s}-e^{-\lambda_{n+1}s})+A_{m,m'}e^{-\lambda_{m'}s}$$ en prenant $m,m'\geq N$ et
en passant aux valeurs absolues on a $$\lvert
    S_{m,m'}\rvert\leq\epsilon(\sum_{n=m}^{m'-1}\lvert e^{-\lambda_n
        s}-e^{-\lambda_{n+1}s}\rvert+1)$$
enfin on applique le lemme 1.0.5 à chaque terme de la somme ce qui nous donne
$$\lvert S_{m,m'}\rvert\leq\epsilon(\frac{\lvert
        s\rvert}{Re(s)}\sum_{n=m}^{m'-1}\lvert e^{-\lambda_n
            Re(s)}-e^{-\lambda_{n+1}Re(s)}\rvert+1)$$ puis $$\lvert
    S_{m,m'}\rvert\leq\epsilon(k(e^{-\lambda_nm}-e^{-\lambda_nm'})+1)$$
Et enfin $$\lvert S_{m,m'}\rvert\leq\epsilon(k+1)$$
qui ne dépend plus de $s$ dans le domaine considéré d'où la convergence
uniforme sur ce domaine puis sur $D(s_0)$.\qed \\
\indent \\

\noindent Dans la suite on laissera le cas général pour se concentrer sur le
cas particulier des séries de Dirichlet proprement dites qui correspond au cas
où $\lambda_n=\log(n)$, $f(s)$ aura donc la forme $\sum \frac{a_n}{n^s}$.

\begin{prop}
    Si $(a_n)$ est bornée, $f$ converge absolument et est holomorphe sur $D(1)$.
\end{prop}

\noindent Pour montrer cette proposition, on a besoin du lemme suivant
:\begin{lem}
    Soit $U$ un ouvert de $\C$, et soit $(f_n)$ une suite de fonctions holomorphes
    sur $U$ qui converge uniformément sur tout compact vers une fonction $f$. La
    fonction $f$ est alors holomorphe dans U.
\end{lem}

{\bf \noindent Preuve du lemme :} Soit $D$ un disque fermé contenu dans $U$, et
soit $C$ son bord orienté dans le sens direct. D'après la formule de Cauchy, on
a $$f_n(s_0)=\int_C\dfrac{f_n(s)}{s-s_0}ds $$ pour tout $s_0\in \mathring{D}$.
Comme $D$ est compact, par l'hypothèse de convergence uniforme on obtient en
passant à la limite $$f(s_0)=\int_C\dfrac{f(s)}{s-s_0}ds $$ d'où l'holomorphie
de $f$ sur $\mathring{D}$ puis sur $U$.\qed \newline

{\bf \noindent Preuve de la propostion :} Montrons que
$\sum_{n\geq1}\frac{a_n}{n^s}$ converge absolument. Si les $a_n$ sont majorés
par $M\in \R$ on a $$\sum_{n\geq1}\mid\frac{a_n}{n^s}\mid\leq
    M.\sum_{n\geq1}\frac{1}{n^{\alpha}}$$
où le terme de droite est une série de Riemann de paramètre $\alpha=Re(s)$,
d'où la convergence absolue pour $s\in D(1)$.\qed \newline

\begin{prop}
    Si les $A_{m,p}=\sum_{n=m}^pa_n$ sont bornées, $f$ converge et est holomorphe
    sur $D(0)$.
\end{prop}{\bf \noindent  Preuve :} Supposons que pour tout $m,p\in\N$ on ait
$\lvert A_{m,p}\rvert\leq K$. En appliquant le lemme d'Abel, on obtient
$$\lvert S_{m,m'}\rvert\leq K(\sum_{m}^{m'-1}\lvert
    \frac{1}{n^s}-\frac{1}{(n+1)^s}\rvert+\lvert\frac{1}{m'^s}\rvert)$$ il suffit
de montrer la convergence pour $s$ réel, alors le résultat sera une conséquence
de la proposition 1.1.2. Supposons donc $s$ réel, on peut enlever les valeurs
absolues et simplifier l'expression ce qui donne $$\lvert S_{m,m'}\rvert\leq
    K((\frac{1}{m^s}-\frac{1}{(m')^s})+\frac{1}{m'^s})$$ puis $$\lvert
    S_{m,m'}\rvert\leq\frac{K}{m^s}$$ d'où la convergence en passant à la limite
$m'\longrightarrow +\infty$.\qed

\begin{defn}
    Une fonction $g~:\N^*\longrightarrow\C$ est dite multiplicative si $$g(1)=1$$
    et $$g(nm)=g(n)g(m)$$ dès que les entiers $n$ et $m$ sont premiers entre eux.
\end{defn}
\noindent Soit $g$ une fonction multiplicative et bornée. On suppose à partir
de maintenant que $a_n=g(n)$, $f(s)$ est donc de la forme $\sum
    \frac{g(n)}{n^s}$, elle converge absolument et est holomorphe sur $D(1)$ par
les résultats précédents.\begin{lem}
    Dans $D(1)$, $f(s)$ est égale au produit infini $$\prod_{p\in
        P}(~\sum_{m\geq0}g(p^m)p^{-ms}~)$$
\end{lem}{\bf \noindent Preuve :} Soit $S$ un sous-ensemble fini de $P$ et soit
$N(S)$ l'ensemble des entiers non nuls dont tous les facteurs premiers sont
dans $S$. Chaque élément $n\in N(S)$ s'écrit alors $n=\prod_{p\in
        S}p^{\alpha_p}$ et on a $$\dfrac{g(n)}{n^s}=\prod_{p\in
        S}g(p^{\alpha_p})p^{-\alpha_p s}$$ d'où en développant $$\prod_{p\in
    S}(~\sum_{m\geq0}g(p^m)p^{-ms}~)$$ on voit que $\dfrac{g(n)}{n^s}$ apparaît une
et une seule fois pour chaque $n\in N(S)$ et la somme en résultant est
$$\sum_{n\in N(S)}\dfrac{g(n)}{n^s}$$ ce résultat étant vrai pour tout
$S\subset ¨P$, lorsque $S$ tend vers $P$ le produit tend alors vers $f(s)$ qui
est le résultat voulu.\qed

\begin{lem}
    Si maintenant $g$ vérifie $g(nm)=g(n)g(m)$ pour chaque paire d'entiers $(n,m)$,
    on a : $$f(s)=\prod_{p\in P}\dfrac{1}{1-g(p)p^{-s}}$$
    Il s'agit du produit eulérien de $f$.
\end{lem}{\bf \noindent Preuve :} Pour tout $m\geq0$ et tout $p\in P$ on a :
$$g(p^m)p^{-ms}=(g(p)p^{-s})^m$$ la série $\sum_{m\geq0}g(p^m)p^{-ms}$ est
alors une série géométrique dont la somme vaut $\dfrac{1}{1-g(p)p^{-s}}$, d'où
le résultat.\qed\\

\noindent On introduit maintenant la fonction zêta de Riemann.\begin{defn}
    On appelle fonction zêta de Riemann notée $\zeta$ la somme de la série
    $\sum_{n\geq1}\frac{1}{n^s}$.
\end{defn}
\noindent On a vu qu'elle est bien définie et holomorphe sur $D(1)$, en effet
on est dans le cas où $g\equiv 1$ qui est multiplicative au sens strict et
bornée.
\begin{prop}
    (Produit Eulérien) Sur $D(1)$ on a :$$\zeta(s)=\prod_{p\in
            P}\dfrac{1}{1-p^{-s}}$$
\end{prop} {\bf \noindent Preuve :} On applique directement le lemme
1.1.8..\qed

\begin{prop}
    On a :$$\zeta(s)=\frac{1}{s-1}+\Phi(s)$$ où $\Phi$ est holomorphe sur $D(0)$.
\end{prop}{\bf \noindent Preuve :} On remarque que
$$\frac{1}{s-1}=\int_1^{+\infty}t^{-s}dt=\sum_{n=1}^{+\infty}\int_{n}^{n+1}t^{-s}dt$$
On peut donc écrire
$$\zeta(s)=\frac{1}{s-1}+\sum_{n=1}^{+\infty}\Big(\frac{1}{n^s}-\int_{n}^{n+1}t^{-s}dt\Big)$$
$$\quad\quad~
    =\frac{1}{s-1}+\sum_{n=1}^{+\infty}\int_{n}^{n+1}(n^{-s}-t^{-s})dt$$ On pose
alors $$\Phi_n(s)=\int_{n}^{n+1}(n^{-s}-t^{-s})dt\quad \text{et}\quad
    \Phi(s)=\sum_{n=1}^{+\infty}\Phi_n(s)$$ On voudrait maintenant appliquer le
lemme 1.1.4, chaque $\Phi_n$ est holomorphe sur $D(0)$ car $s\longmapsto
    n^{-s}-t^{-s}$ est holomorphe pour tout $t\in[n,n+1]$. Montrons donc que
$\sum_{n\geq1}\Phi_n(s)$ converge normalement sur tout compact de $D(0)$. Soit
$K$ un compact de $D(0)$, on a : $$\lvert \Phi_n(s)\rvert\leq \sup_{s\in
        K}\Big(\sup_{t\in[n,n+1]}\lvert n^{-s}-t^{-s}\rvert\Big)$$ mais
$(n^{-s}-t^{-s})$ a pour dérivée $\dfrac{s}{t^{s+1}}$ de valeur absolue
strictement monotone pour $t$ et atteignant son maximum en $t=n$, d'où $$\lvert
    \Phi_n(s)\rvert\leq\sup_{s\in K}\Big(\dfrac{\lvert s\rvert}{n^{Re(s)+1}}\Big)$$
puis la convergence normale découle du fait que $\Phi(s)$ est alors majorée par
une série de Riemann de paramètre $Re(s)+1>1$ d'où le résultat.\qed
\begin{cor}On a :
    $$\zeta(s)\sim_{1} \dfrac{1}{s-1}$$
\end{cor}{\bf \noindent Preuve :}Par la proposition précédente $\Phi$ reste
bornée dans un voisinage de $1$, d'où le résultat.\qed\\\indent\\ À partir de
maintenant on appellera par $\log$ la branche principale du logarithme qui peut
être définie par
$\log\Big(\dfrac{1}{1-\alpha}\Big)=\sum_{n=1}^{+\infty}\dfrac{\alpha^n}{n}$
pour $\lvert\alpha\rvert<1$. Alors $\log$ est bien définie sur $D(0)$,
$\log(1)=0$ et on peut vérifier que
$$\log\Big(\dfrac{1}{(1-\alpha)(1-\beta)}\Big)=\log\Big(\dfrac{1}{1-\alpha}\Big)+\log\Big(\dfrac{1}{1-\beta}\Big)$$
pour $\lvert\alpha\rvert,\lvert\beta\rvert<1$.
\begin{cor}
    Lorsque $s\rightarrow 1$ on a :$$\sum_{p\in P}p^{-s}\sim
        \log\Big(\frac{1}{s-1}\Big)$$ tandis que $$\sum_{p\in
            P,n\geq1}\frac{(p^{-s})^n}{n}$$ reste bornée.
\end{cor}{\bf \noindent Preuve :} Soit $s\in D(1)$, on va faire tendre $s$ vers
$1$ dans $D(1)$ donc on peut le supposer dans un petit voisinage de $1$. La
quantité $\log\dfrac{1}{s-1}$ est alors bien définie et on a par le produit
Eulérien de $\zeta$ (proposition 1.1.10): $$\log\zeta(s)=\log\Big(\prod_{p\in
        P}\dfrac{1}{1-p^{-s}}\Big)$$ $$\qquad\qquad\quad=\sum_{p\in
        P}\log\Big(\dfrac{1}{1-p^{-s}}\Big)$$ $$~\quad\qquad=\sum_{p\in
        P,n\geq1}\frac{(p^{-s})^n}{n}$$ $$\qquad\qquad\qquad\qquad=\sum_{p\in
        P}p^{-s}~+\sum_{p\in P,n\geq2}\frac{(p^{-s})^n}{n}$$ comme $\log\zeta(s)\sim_1
    \log(\frac{1}{s-1})$ il suffit de montrer que $\sum_{p\in
        P,n\geq2}\frac{(p^{-s})^n}{n}$ reste bornée dans un voisinage de $1$ alors on
aura $\sum_{p\in P}p^{-s}\sim_1 \log(\frac{1}{s-1})$ qui est le résultat voulu.
Mais on a les majorations suivantes $$\sum_{p\in
        P,n\geq2}\left|\frac{(p^{-s})^n}{n}\right|\leq\sum_{p\in
        P,n\geq2}\left|(p^{-s})^n\right|$$ $$\qquad\qquad~\leq\sum_{p\in
        P}\left|p^{-2s}\right|$$ $$\qquad\qquad\qquad\qquad\leq\sum_{p\in
        P}\left|p^{-s}(p-1)^{-s}\right|$$ $$\qquad\qquad\qquad\qquad\leq\sum_{p\in
        P}p^{-1}(p-1)^{-1}$$
$$\qquad\qquad\qquad\qquad\qquad\leq\sum_{n=2}^{+\infty}n^{-1}(n-1)^{-1}=1$$ en
effet la série $\sum_{n\geq2}n^{-1}((n-1)^{-1})=\sum_{n\geq1}(n+1)^{-1}n^{-1}$
est convergente et le calcul s'effectue en remarquant que
$$(n+1)^{-1}n^{-1}=n^{-1}-(n+1)^{-1}$$ d'où le résultat.\qed\\
\indent\\\noindent Cet équivalent permet alors de donner du sens à la
définition de densité introduite en début de partie. Il existe en fait une
notion de densité dite naturelle que l'on définit de la manière suivante : Si
$A$ est un ensemble de nombre premier, on définit la densité naturelle de $A$
comme la limite de $\dfrac{\lvert A\cap[1,n]\rvert}{P\cap[1,n]}$ lorsque
$n\rightarrow \infty$. Même si celle-ci semble être la "bonne" notion de
densité, elle n'existe pas toujours et est bien plus difficile à calculer que
celle de Dirichlet. De plus on peut montrer que si celle-ci existe elle
coïncide avec la densité de Dirichlet.

\subsection{Fonctions L de Dirichlet}
Soit $m\geq1$ un entier. On note $G(m)$ le groupe des inversibles de $\Z/m\Z$
et $\widehat{G(m)}$ son dual, qui correspond à l'ensemble des caractères de
degré $1$ sur $G(m)$. $G(m)$ est un groupe abélien fini d'ordre $\phi(m)$ (où
$\phi$ désigne l'indicatrice d'Euler). On note $\overline{n}$ la classe de
$n\in\Z$ dans $G(m)$. \begin{defn}On appelle caractère modulo $m$ un caractère
    $\chi$ de $G(m)$ que l'on étend à $\Z$ en posant $$\chi(n)=\chi(\overline{n})
        \quad\text{si } n\text{ est premier à }m,~\chi(n)=0\quad\text{sinon}.$$
\end{defn}

\noindent On peut maintenant introduire les fonctions $L$ de Dirichlet. Soit
$\chi$ un caractère modulo $m$.
\begin{defn}
    On appelle fonction L de Dirichlet correspondant à $\chi$ et $m$ la série de
    Dirichlet définie par $$L(s,\chi):=\sum_{n=1}^{+\infty}\dfrac{\chi(n)}{n^s}$$
\end{defn}

\noindent Remarquons que $L(s,\chi)$ est une série de Dirichlet proprement dîte
telle que $a_n=\chi(n)$ où $\chi$ est multiplicative au sens strict et bornée.
Lorsque $\chi=1$, on a $$L(s,1)=\prod_{p\mid m}(1-p^{-s})\zeta(s)$$ qui est le
produit d'une fonction holomorphe et non nulle sur $D(0)$ avec $\zeta$ qui est,
on l'a vu, méromorphe sur $D(0)$ avec un seul pôle simple en $1$ d'où $L(s,1)$
est méromorphe sur $D(0)$ avec un unique pôle simple en $1$. \\\indent De plus
lorsque $\chi\ne1$ les sommes $A_{u,v}$ (avec les notations du lemme d'Abel)
sont bornées par $\phi(m)$. Montrons le dernier point. On peut supposer
$(v-u)<m$, en effet par la proposition 1.2.2, on a
$$\sum_{n=u}^{u+m-1}\chi(n)=0$$ Si maintenant $x\in \Z$ est premier à $m$ alors
$$\chi(x^{\phi(m)})=\chi(x)^{\phi(m)}=\chi(\overline{x}^{\phi(m)})=\chi(1)=1$$
d'où $\chi(x)$ est une racine $\phi(m)$-ème de l'unité. En particulier
$\left|\chi(x)\right|\leq1$ pour tout $x\in \Z$ d'où
$$\left|\sum_{n=u}^{v}\chi(n)\right|\leq\sum_{n=u}^{v}\left|\chi(n)\right|\leq\phi(m)$$
qui est le résultat voulu. Par les propositions 1.1.3., 1.1.5. et le lemme
1.1.8. $L(s,\chi)$ converge absolument et est holomorphe sur $D(1)$, converge
et est holomorphe sur $D(0)$ et dans $D(1)$ on a : $$L(s,\chi)=\prod_{p\in
        P}\dfrac{1}{1-\chi(p)p^{-s}}$$ Dans la suite, si $p$ est premier et ne divise
pas $m$ on notera $o(p)$ l'ordre de $\overline{p}$ dans $G(m)$ et
$q(p)=\phi(m)/o(p)$ qui est l'ordre du groupe quotient
$G(m)/<\overline{p}>$.\begin{lem}
    Si $p$ ne divise pas $m$, on a l'identité
    $$\prod_{\chi\in\widehat{G(m)}}(1-\chi(p)T)=(1-T^{o(p)})^{q(p)}$$
\end{lem}{\bf \noindent Preuve :} Soit $U_{k}$ l'ensemble des racines $k$-èmes
de l'unité, on a : $$\prod_{w\in U_{o(p)}}(1-wT)=1-T^{o(p)}$$ de plus on sait
par la théorie des caractères que l'opération de restriction de
$\chi\in\widehat{G(m)}$ dans $\widehat{<\overline{p}>}$ est un homomorphisme
surjectif de noyau les caractères triviaux sur $<\overline{p}>$, d'où la suite
exacte courte
$$\{1\}\rightarrow\widehat{\dfrac{G(m)}{<\overline{p}>}}\rightarrow\widehat{G(m)}\rightarrow\widehat{<\overline{p}>}\rightarrow{1}$$Et
il existe un unique caractère $\chi_1$ de $<\overline{p}>$ tel que
$\chi_1(p)=w$ pour chaque $w\in U_{o(p)}$ que l'on peut étendre en
$\overline{\chi_1}$ un caractère de $G(m)$. Enfin l'application
$\chi\longrightarrow \chi.\overline{\chi_1}^{-1}$ est une bijection de
$\widehat{G(m)}$ dans lui même et envoie le noyau de la restriction sur
l'ensemble des caractères $\chi$ de $G(m)$ tels que $\chi(\overline{p})=w$.
D'où il existe exactement $q(p)$ caractères tels que $\chi(\overline{p})=w$. En
mettant en commun les deux résultats on obtient directement le résultat
voulu.\qed \\\indent\\

\noindent On définit maintenant la fonction $\zeta_m(s)$ comme le produit
$$\zeta_m(s):=\prod_{\chi\in \widehat{G(m)}}L(s,\chi)$$On a la proposition
suivante :\begin{prop}
    On a :$$\zeta_m(s)=\prod_{p\nmid m}\dfrac{1}{(1-\dfrac{1}{p^{o(p)s}})^{q(p)}}$$
    C'est une série de Dirichlet à coefficients entiers positifs, convergeant dans
    $D(1)$.
\end{prop}{\bf \noindent Preuve :} On va appliquer le lemme 1.2.3. avec
$T=p^{-s}$, rappelons que $\chi(p)=0$ lorsque $p\mid m$, on a :$$\prod_{\chi\in
        \widehat{G(m)}}L(s,\chi)=\prod_{\chi\in \widehat{G(m)}}\big(\prod_{p\in
        P}\dfrac{1}{1-\chi(p)p^{-s}}\Big)$$ $$~~\qquad\quad\qquad=\prod_{p\in
        P}\Big(\prod_{\chi\in \widehat{G(m)}}\dfrac{1}{1-\chi(p)p^{-s}})$$
$$~\qquad\quad=\prod_{p\nmid m}\dfrac{1}{(1-p^{-o(s)s})^{q(p)}}$$Qui est le
résultat voulu. Le produit converge dans $D(1)$ comme produit de séries de
Dirichlet convergentes dans $D(1)$.\\ \qed

\noindent En développant le produit on voit que $\zeta_m$ est une série de
Dirichlet à coefficients positifs puis qu'elle est holomorphe sur $D(1)$. Cette
remarque est importante car pour prouver le prochain théorème on va avoir
besoin du lemme suivant sur les séries de Dirichlet à coefficients positifs.
\begin{lem}
    Soit $f$ une série de Dirichlet à coefficients positifs. Supposons que $f$
    converge sur $D(\rho)$ et qu'elle est prolongeable analytiquement en un
    voisinage de $\rho$. Alors il existe $\epsilon>0$ tel que $f$ converge sur
    $D(\rho-\epsilon)$.
\end{lem}
{\bf \noindent Preuve :} Pour une preuve du lemme, on pourra se référer à [1]
p.112.\qed

\indent \newline La première étape de la preuve du théorème qui nous interesse
consiste en le théorème suivant :

\begin{thm}
    \begin{itemize}\indent\\
        \item $\zeta_m$ a un pôle simple pour $s=1$.
        \item $L(1,\chi)\ne0$ pour tout $\chi\ne1$
    \end{itemize}
\end{thm}{\bf \noindent Preuve :} Remarquons que le second point implique le
premier. En effet on a vu en début de partie que $L(s,1)$ a un pôle en $s=1$,
alors si $L(1,\chi)\ne0$ pour $\chi\ne1$, $\zeta_m$ à un pôle en $s=1$.
Montrons donc le second point. On procède par l'absurde, supposons que pour un
$\chi\ne1$ on ait $L(1,\chi)=0$. La fonction $\zeta_m$ serait alors holomorphe
en $s=1$, comme les $L(s,\chi)$ sont holomorphes sur $\D(0)\backslash\{1\}$,
$\zeta_m$ serait holomorphe sur $D(0)$. Par le lemme 1.2.5. on peut donc
reculer son domaine de convergence à $D(0)$. On considère maintenant $p\in P$,
on a
$$\sum_{n=0}^{+\infty}p^{-n\phi(m)s}=\sum_{n=0}^{+\infty}p^{-no(p)q(p)s}\leq\Big(\sum_{n=0}^{+\infty}p^{-no(p)s}\Big)^{q(p)}=\dfrac{1}{(1-p^{-o(p)s})^{q(p)}}$$
$\zeta_m$ étant le produit du deuxième terme pour tout $p\nmid m$, par le même
procédé de majoration la série $\zeta_m$ aurait tout ses coefficients
supérieurs à ceux de la série : $$\sum_{(n,m)=1}n^{-\phi(m)s}$$ qui diverge
pour $s=1/\phi(m)$ d'où le résultat voulu.\qed

\subsection{Le théorème de Dirichlet}
Comme vu en début de chapitre, le but de cette partie sera de déterminer un
équivalent de $g_a(s):=\sum_{p\in P_a}p^{-s}$ pour $s\rightarrow 1$ et ainsi
déterminer la densité de $P_a$. Soit $\chi$ un caractère modulo $m$. On pose
$$f_{\chi}(s)=\sum_{p\nmid m}\chi(p)p^{-s}$$\begin{lem}
    $f_{\chi}$ converge sur $D(1)$ et on a:\begin{itemize}
        \item Si $\chi=1$, $f_\chi\sim_1 log(1/(s-1))$.
        \item Si $\chi\ne1$, $f_{\chi}$ reste bornée lorsque $s\rightarrow1$.
    \end{itemize}
\end{lem}{\bf \noindent Preuve :} Le premier point est une conséquence directe
de la première partie du corollaire 1.1.13 car $f_{\chi}(s)$ ne diffère de
$\sum_{p\in P}p^{-s}$ que par un nombre fini de termes. Le second point est une
conséquence de la seconde partie du même corollaire. En effet, on a $$\log
    L(s,\chi)=f_{\chi}(s)+F_{\chi}(s))$$ avec
$F_{\chi}(s)=\sum_{p,n\geq2}\chi(p)^n(np)^{-ns}$ mais
$$\left|F_{\chi}(s)\right|\leq\sum_{p\in P,n\geq2}(np)^{-ns}$$ de plus par le
théorème $1.2.5.b)$ $L(1,\chi)\ne0$ donc $\log L(s,\chi)$ et $F_{\chi}(s)$
restent bornées quand $s\rightarrow1$, il en est donc de même pour
$f_{\chi}(s)$.\qed \begin{lem}
    On a
    :$$g_a(s)=\dfrac{1}{\phi(m)}\sum_{\chi\in\widehat{G(m)}}\chi(a)^{-1}f_{\chi}(s)$$
\end{lem}{\bf \noindent Preuve :} On a
$$\sum_{\chi\in\widehat{G(m)}}\chi(a)^{-1}f_{\chi}(s)=\sum_{p\nmid
        m}\Big(\sum_{\chi\in\widehat{G(m)}}\chi(a^{-1}p)\Big)p^{-s}$$ mais les
relations d'orthogonalité des caractères de $G(m)$ donnent
$$\sum_{\chi\in\widehat{G(m)}}\chi(a^{-1}p)=\phi(m)\quad \text{si}\quad
    a^{-1}p\equiv1\mod m$$ $$=0~\qquad~\text{sinon.}$$ on a donc bien
$$g_a(s)=\dfrac{1}{\phi(m)}\sum_{\chi\in\widehat{G(m)}}\chi(a)^{-1}f_{\chi}(s).$$\qed

On peut maintenant prouver le théorème :\begin{thm}
    L'ensemble $P_a$ a une densité et sa densité vaut $\dfrac{1}{\phi(m)}$.
\end{thm}
\begin{cor}
    L'ensemble $P_a$ est infini.
\end{cor}{\bf \noindent Preuve :} Par le lemme 1.3.2.
$$g_a(s)=\dfrac{1}{\phi(m)}\sum_{\chi\in\widehat{G(m)}}\chi(a)^{-1}f_{\chi}(s)$$
mais par le lemme 1.3.1.
$$\sum_{\chi\in\widehat{G(m)}}\chi(a)^{-1}f_{\chi}(s)\sim \log\dfrac{1}{s-1}$$
pour $s\rightarrow1$ d'où $$g_a(s)\sim \dfrac{1}{\phi(m)}\log\dfrac{1}{s-1}$$
ce qui veut bien dire que $P_a$ a pour densité $\dfrac{1}{\phi(m)}$.\qed

\subsection{Une application du théorème de Dirichlet}
Il existe de nombreuses applications du théorème de la progression arithmétique
de Dirichlet, ici nous traitons un résultat basique de la théorie de Galois
inverse qui se trouve être une conséquence de ce théorème. La question est la
suivante : Soit $G$ un groupe. Existe-t-il une extension galoisienne $L$ de
$\Q$ de groupe de Galois $G$ ? La plus grande conjecture à ce sujet est la
suivante : Tout groupe fini est le groupe de Galois d'une extension galoisienne
des nombres rationnels. On se restreint ici au cas bien plus modeste où $G$ est
abélien fini.

Soit donc $G$ un groupe abélien fini, c'est un $\Z$-module de type
fini. Grâce au théorème de structure sur les $Z$-modules de type fini, on peut
supposer que G est de la forme $$\bigoplus_i \Z_{q_i^{\beta_i}}$$ où les $q_i$
sont des nombres premiers non nécessairement distincts. On veut montrer qu'il
existe une extension galoisienne de groupe de Galois $G$. Il suffit de faire
apparaître $\bigoplus_i \Z_{q_i^{\beta_i}}$ dans $$(\Z_m)^*\cong\Z_n\times
    \prod_{i=1}^{k}Z_{p_i-1}$$ où $n=\prod_{i=1}^{k}p_i^{\alpha_i}$ est un entier
décomposé en produit de facteurs premiers. Comme les $\Z_{p_i-1}$ sont
cycliques il suffit de trouver des $p_i$ tels que $q_i^{\beta_i}$ divise
$p_i-1$ (condition $(*)$)alors $\bigoplus_i \Z_{q_i^{\beta_i}}$ sera un sous
groupe de $\prod_i\Z_{p_i-1}$ correspondant à un groupe de Galois. En effet, si
on considère $\zeta$ une racine primitive $m$-ème de l'unité dans $\C$ alors on
sait par le cours que $\Q(\zeta)/\Q$ a pour groupe de Galois
$G_1\cong(\Z_m)^*$, le théorème de correspondance de Galois fournira alors un
corps $E$ tel que $\Q\subset E\subset \Q(\zeta)$ et $Gal(\Q(\zeta)/E)\cong G$
qui est le résultat voulu. \\ \indent On réécrit la condition $(*)$ en
$$\forall i=1,...,N \quad \exists p_i>q_i^{\beta_i}\quad et\quad  p_i\equiv
    1\mod q_i^{\beta_i}$$ qui est une conséquence immédiate du théorème de la
progression arithmétique de Dirichlet appliqué à $q_i^{\beta_i}$ et $1$ d'où le
résultat.

\section{Anneaux d'entiers de corps de nombres}
On rappelle ici quelques définitions et résultats sans preuve qui nous serons
utiles par la suite. Les anneaux sont supposés commutatifs et unitaires.

\subsection{Anneaux d'entiers}

\begin{defn}
    Soit $R$ un anneau et $A$ un sous-anneau de $R$.\begin{itemize}
        \item On dit que $x\in R$ est entier sur $A$ si il existe $P\in A[X]$
              unitaire tel que $P(x)=0$.
        \item L'ensemble des éléments $x\in R$ entiers sur $A$ forment un sous
              anneau $A'$ de $R$ que l'on appelle fermeture intégrale de $A$ dans $R$.
        \item On dit que $A$ est intégralement clos s'il est intègre et que sa
              fermeture intégrale dans son corps des fractions est $A$ lui-même.
    \end{itemize}
\end{defn}

\begin{prop}Soit $R$ un anneau et $A$ un sous anneau de $R$.
    Les propriétés suivantes sont équivalentes :\begin{itemize}
        \item $x\in R$ est entier sur $A$.
        \item Il existe $B$ un sous $A$-module de type fini de $R$ tel que $x\in
                  B$.
    \end{itemize}

\end{prop}

\begin{prop}
    Soit $R$ un anneau intègre et $A$ un sous-anneau de $R$ tel que $R$ est entier
    sur $A$. Alors $A$ est un corps si et seulement si $R$ est un corps.
\end{prop}

\begin{defn}
    Soit $R$ un anneau et $A$ un sous-anneau de $R$ tel que $R$ est un $A$-module
    libre de rang $n$.\begin{itemize}
        \item On appelle trace de $x\in R$ la trace de la multiplication par $x$ en
              tant qu'endomorphisme du $A$-module $R$. On la note $Tr_{R/A}(x)$ ou $Tr(x)$
              lorsque le contexte est clair.
        \item On appelle norme de $x\in R$ le déterminant de la multiplication par
              $x$ en tant qu'endomorphisme du $A$-module $R$. On la note $N_{R/A}(x)$ ou
              $N(x)$ lorsque le contexte est clair.
        \item Pour toute famille $(x_1,...,x_n)\in R^n$ on définit le discriminant
              de cette famille par $$D(x_1,...,x_n)=det((Tr(x_ix_j))_{1\leq i,j\leq n})$$
    \end{itemize}
\end{defn}

\begin{prop}
    Soient $K/L$ une extension séparable de degré $n$, $x$ un élément de $L$ et $P$
    son polynôme minimal dans $K$. La trace et la norme de $x$ sont entières sur
    $A$ et en notant $x_1,...,x_n$ les racines de $P$ on a $$Tr(x)=x_1+...+x_n$$ et
    $$N(x)=x_1...x_n$$
\end{prop}

\noindent La proposition 2.1.6. que l'on montre à l'aide de 2.1.5. sera utile
lors du prochain chapitre.
\begin{prop}
    Dans les mêmes conditions que précédemment. En notant $\sigma_1,...,\sigma_n$
    les $n$ $K$-plongements de $L$ dans une de ses clôtures algébriques, on a pour
    toute $K$-base $(x_1,...,x_n)$ de $L$
    $$D(x_1,...,x_n)=det((\sigma_i(x_j))_{1\leq i,j\leq n})\ne0.$$
\end{prop}

\begin{defn}
    Soit $R$ un anneau et $A$ un sous-anneau de $R$ tel que $R$ est un $A$-module
    libre de rang $n$. On appelle discriminant de $R$ sur $A$ et note $\mathfrak
        D_{R/A}$ l'idéal engendré par le discriminant d'une $A$-base de $R$.
\end{defn}

\noindent Le discriminant de $R$ sur $A$ est bien défini car on montre par un
bref calcul que si $(x_1,...,x_n)$ et $(y_1,...y_n)$ sont deux $A$-bases de $R$
alors $D(x_1,...,x_n)=p^2D(y_1,...,y_n)$ où $p$ est le déterminant de la
matrice de passage de $(x_1,...,x_n)$ à $(y_1,...,y_n)$ alors $p\in A^{\times}$
et ils définissent donc le même idéal.

\begin{thm}
    Soient $A$ un anneau intégralement clos de caractéristique $0$, $K$ son corps
    des fractions, $L$ une extension finie de degré $n$ de $K$ et $R$ la fermeture
    intégrale de $A$ dans $L$. Alors $R$ est un sous $A$-module d'un $A$-module
    libre de rang $n$ et contient une base de $L$ sur $K$.
\end{thm}

\subsection{Modules noethérien et anneau de Dedekind}
\begin{defn}
    On dit qu'un module est noethérien si tout ses idéaux sont de types fini.
\end{defn}

\noindent En particulier on dit qu'un anneau est noethérien si ses idéaux sont
de type finis.

\begin{prop}
    Si $A$ est un anneau noethérien et $E$ un $A$-module de type fini alors $E$
    est noethérien.
\end{prop}

\begin{lem}
    Dans un anneau noethérien, tout idéal contient un produit d'idéaux premiers.
\end{lem}

\begin{defn}
    On appelle anneau de Dedekind tout anneau noethérien, intégralement clos dont
    les ideaux premiers non nuls sont maximaux.
\end{defn}

\begin{thm}
    Soient $A$ un anneau de Dedekind de caractéristique $0$, $K$ son corps des
    fractions, $L$ une extension finie de $K$ et $R$ la fermeture intégrale de $A$
    dans $L$. Alors $B$ est un anneau de Dedekind et un $A$-module de type fini.
\end{thm}

\begin{defn}
    Soient $A$ un anneau de Dedekind, $K$ son corps des fractions. On appelle idéal
    fractionnaire de $A$ tout $A$-module de type fini contenu dans $K$. On note
    $I(A)$ l'ensemble des idéaux fractionnaires de $A$. On dit de plus qu'un idéal
    fractionnaire est entier s'il est contenu dans $A$. On note de plus $P(A)$
    l'ensemble des idéaux fractionnaires principaux de $A$.
\end{defn}

\noindent On appelle $\mathfrak A(A)$ l'ensemble des idéaux premiers de $A$.

\begin{thm}
    Soient $A$ un anneau de Dedekind. L'ensemble $ I(A)\backslash\{0\}$ est un
    groupe pour la multiplication usuelle des idéaux. De plus, tout idéal
    fractionnaire non nul $I$ de $A$ s'écrit de manière unique sous la forme
    $$I=\prod_{\mathfrak p\in \mathfrak A(A)}\mathfrak p^{n_{\mathfrak p}(I)}$$ où
    les $n_{\mathfrak p}(I)$ sont des entiers relatifs presque tous nuls.
\end{thm}

\noindent Dans le cas des corps de nombre, l'anneau des entiers de ces corps
sont par définition des fermetures intégrales de $\Z$ dans des extensions
finies de $\Q$. Ce sont donc des anneaux de Dedekind par le théorème 3.2.2.
donc le théorème 3.2.4. s'applique.

\section{Discriminant et ramification}
Dans la suite $A$ est un anneau intègre de caractéristique 0, $K$ son corps des
fractions, $L$ une extension de degré $n$ de $K$ et $R$ la fermeture intégrale
de $A$ dans $L$.
\subsection{Localisation dans un anneau de Dedekind}
On rappelle ici quelques résultats sur la localisation dans les anneaux de
Dedekind.

\begin{prop}
    Soit $S$ une partie multiplicative de $A$. Alors $S^{-1}R$ est la fermeture
    intégrale de $S^{-1}A$ dans $L$.
\end{prop}
\begin{prop}
    On suppose que $A$ est un anneau de Dedekind. Alors, $S^{-1}A$ est un anneau de
    Dedekind.
\end{prop}
\begin{prop}
    Soit $\mathfrak p $ un idéal premier non nul de $A$ et $S=A\backslash \mathfrak
        p$. Alors $S^{-1}A$ est principal et il existe $p$ premier de $S^{-1}A$ tel que
    les idéaux de $S^{-1}A$ sont exactement les $(p^n)n\geq1$.
\end{prop}
\noindent La prochaine proposition rend compte de l'utilité de la localisation
dans les anneaux de Dedekind, en effet en combinant celle ci avec 3.1.3. on
pourra toujours se ramener au cas principal ce qui sera particulièrement utile
dans la prochaine partie.
\begin{prop}
    Soit $S$ une partie multiplicative de $A$ et $\mathfrak m$ un idéal maximal de
    $A$ tel que $\mathfrak m\cap S=\emptyset$. Alors $$S^{-1}A/\mathfrak
        mS^{-1}A\cong A/\mathfrak m$$
\end{prop}
\subsection{Décomposition des idéaux premiers dans une extension}
Soit $A$ un anneau de Dedekind de caractéristique $0$, $K$ son corps des
fractions, $L$ une extension de degré $n$ de $K$ et $R$ la fermeture intégrale
de $A$ dans $L$. Par le théorème 2.2.5., $R$ est un anneau de Dedekind. Soit
$\mathfrak p$ un idéal premier non nul de $A$. On peut décomposer l'idéal
$R\mathfrak p$ en produits d'idéaux premiers entiers de $R$
$$\prod_{i=1}^q\mathfrak P_i^{e_i}$$ avec $e_i\geq1$ pour chaque $i$.

\begin{prop}
    Si $\mathfrak D$ est un idéal premier de $R$ tel que $R\mathfrak
        p\mathfrak\subset\mathfrak D$ alors $\mathfrak D=\mathfrak P_i$ pour un certain
    $i$.
\end{prop}

{\noindent \bf Preuve :} Soit $\mathfrak P$ un idéal premier de $R$. La
condition $R\mathfrak p\subset \mathfrak D$ implique $R\mathfrak p\mathfrak
    D^{-1}\subset A$ d'où $n_{\mathfrak P}(R\mathfrak p)\geq\mathfrak n_{\mathfrak
            P}(\mathfrak D)\geq 0$ pour tout $\mathfrak P\in \mathfrak A(R)$. Le résultat
en découle directement. \qed \\

\noindent Notons que si $R\mathfrak p\subset \mathfrak D$ alors $R\mathfrak
    p\cap A\subset \mathfrak D\cap A$ or $R\mathfrak p\cap A=\mathfrak p$ est
maximal donc $\mathfrak p=\mathfrak D\cap A$. D'où $$ A\rightarrow R\rightarrow
    R/\mathfrak P_i$$ a pour noyau $\mathfrak p$ donc $A/\mathfrak p$ s'identifie à
un sous-anneau de $R/\mathfrak P_i$. Enfin par 3.2.2. ces deux anneaux sont des
corps et $R$ est un $A$-module de type fini donc $R/R\mathfrak p$ est un
$A/\mathfrak p$-espace vectoriel de dimension finie. On définit maintenant deux
nouvelles quantités.

\begin{defn}
    On appelle degré résiduel de $\mathfrak P_i$ sur $A$ la dimension du
    $A/\mathfrak p$-espace vectoriel $R/\mathfrak P_i$.

\end{defn}

\begin{defn}
    On appelle indice de ramification de $\mathfrak P_i$ sur $A$ l'exposant $e_i$
    de la décomposition de $R\mathfrak p$ dans $R$.
\end{defn}

\noindent Enfin, remarquons que $R\mathfrak p\cap A=\mathfrak p$. En effet
l'inclusion $\supset$ est claire. D'un autre côté $R\mathfrak p\cap A$ est un
idéal de $A$ qui contient un idéal maximal, comme $1\notin R\mathfrak p$,
$\subset$ est vraie. D'où $R/R\mathfrak p$ est un $A/\mathfrak p$-espace
vectoriel de dimension finie par le même argument que précédemment. Pour $K$ un
corps et $L$ une extension finie de $K$ on note $[L:K]$ la dimension de $L$ en
tant que $K$-espace vectoriel.

\begin{thm}
    Avec les notations précédentes on a $\sum_{i=1}^q e_if_i=[R/R\mathfrak p :
        A/\mathfrak p]=n$
\end{thm}
En fait, on verra dans la prochaine partie que si $L/K$ est galoisienne, alors
$e_1=e_2=...=e_q$ et $f_1=f_2=...=f_q$ d'où $n=efq$ en notant $e=e_1$ et
$f=f_1$.

    {\bf \noindent Preuve :} Pour la première égalité, remarquons que tout quotient
de la forme $$\mathfrak B/\mathfrak B\mathfrak P_i,$$ où $\mathfrak B$ est un
produit d'idéaux de la forme $\mathfrak P_1^{e_1}...\mathfrak
    P_{i-1}^{e_{i-1}}\mathfrak P_{i}^{k}$ avec $0\leq k\leq q$, est un $R$-module
annulé par $\mathfrak P_i$ d'où $\mathfrak B/\mathfrak B\mathfrak P_i$ est un
$R/\mathfrak P_i$-espace vectoriel et on a l'isomorphisme d'espaces vectoriels
suivant $$R/R\mathfrak p\cong R/\mathfrak P_1\bigoplus \mathfrak P_1/\mathfrak
    P_1^2\bigoplus...\bigoplus \mathfrak P_1^{e_1}/\mathfrak P_1^{e_1}\mathfrak
    P_2\bigoplus...\bigoplus \mathfrak P_1^{e_1}...\mathfrak P_q^{e_q-1}/\mathfrak
    P_1^{e_1}...\mathfrak P_q^{e_q}$$ en plus $\mathfrak B/\mathfrak B\mathfrak
    P_i$ est un espace vectoriel de dimension $1$ sur $R/\mathfrak P_i$ car ses
sous espaces vectoriels sont des sous $R$-modules de $\mathfrak B/\mathfrak
    B\mathfrak P_i$ et donc des ensembles de la forme $\mathfrak q/\mathfrak
    B\mathfrak P_i$ avec $\mathfrak q$ un idéal de $R$ tel que $\mathfrak
    B\mathfrak P_i\subset \mathfrak q\subset \mathfrak B$ mais la décomposition en
idéaux premiers dans $R$ implique que $\mathfrak q=\mathfrak B\mathfrak P_i$ ou
$\mathfrak q=\mathfrak B$ d'où le résultat. Enfin $R/\mathfrak P_i$ est de
dimension $f_i$ sur $A/\mathfrak p$ et il y a $e_i$ espaces vectoriels de la
forme $\mathfrak B/\mathfrak B\mathfrak P_i$ d'où le résultat. \\ \indent Pour
la deuxième égalité on se place d'abord dans le cas ou $A$ est principal. Par
le théorème 3.1.6. $R$ est alors un $A$-module libre de rang $n$ et une
$A$-base de $R$ donne par réduction modulo $R\mathfrak p$ une $A/\mathfrak
    p$-base de $R/R\mathfrak p$. On se ramène à ce cas en localisant $A$ et $R$ en
$S=A\backslash\mathfrak p$. On note $A'=S^{-1}A$ et $R'=S^{-1}R$ les anneaux de
fractions correspondants. On sait que $A'$ est un anneau principal possédant un
unique idéal maximal $A'\mathfrak p$ et que $R'$ est la fermeture intégrale de
$A'$ dans $L$. On a alors $[R'/R'\mathfrak p:A'/A'\mathfrak p]=n$ par le cas
principal. De plus $R'$ est un anneau de Dedekind, $R'\mathfrak p$ se décompose
donc en idéaux premiers. Remarquons que par 4.2.1. $\mathfrak P_i\cap
    A\backslash\mathfrak p=\emptyset$ donc par 4.1.1. $R'\mathfrak P_i$ est un
idéal premier non nul de $R'$. Enfin la décomposition de $R\mathfrak p$ dans
$R$ donne directement $$R'\mathfrak p=\prod_{i=1}^q(R'\mathfrak P_i)^{e_i}.$$
On applique alors la première partie de la preuve à $A'$ et $B'$ ce qui donne
$$n=[R'/R'\mathfrak p:A'/A'\mathfrak p]=\sum_{i=1}^qe_i[R'/R'\mathfrak
    P_i:A'/A'\mathfrak p]$$ mais $A'/A'\mathfrak p\cong A/\mathfrak p$ et
$R'/R'\mathfrak P_i\cong R/\mathfrak P_i$ d'où $[R'/R'\mathfrak
            P_i:A'/A'\mathfrak p]=f_i$ pour chaque $i$ puis le résultat voulu.\qed \\

\begin{prop}
    On a l'isomorphisme $$R/R\mathfrak p\cong \prod_{i=1}^qR/\mathfrak P_i^{e_i}$$
\end{prop}

\noindent Il suffit d'appliquer le lemme chinois sachant que $\mathfrak
    P_i^{e_i}+\mathfrak P_j^{e_j}=R$ dès que $i\ne j$, en effet il est clair que le
seul idéal maximal contenant $\mathfrak P_i^{e_i}$ est $\mathfrak P_i$. On
rappelle l'énoncé du lemme.

\begin{lem}
    Soit $A$ un anneau, et $(\mathfrak a_i)_{i=1,...,q}$ une famille finie d'idéaux
    de $A$ tels que $\mathfrak a_i+\mathfrak a_j=A$ dès que $i\ne j$. On a alors
    l'isomorphisme $$A/\mathfrak a_1...\mathfrak a_q\cong \prod_{i=1}^r A/\mathfrak
        a_i$$
\end{lem}

\subsection{Discriminant et Ramification}
On garde les notations de 3.2.

\begin{defn}
    On dit que $\mathfrak p$, en tant qu'idéal de $A$, se ramifie dans $R$ (ou dans
    $L$) si l'un des indices de ramification $e_i$ est plus grand que 1.
\end{defn}

\begin{defn}
    On suppose que $K$ et $L$ sont des corps de nombres. On appelle idéal
    discriminant de $R$ sur $A$ (ou de $L$ sur $K$), et on note $\mathfrak O_{R/A}$
    (ou $\mathfrak O_{L/K})$ l'idéal de $A$ engendré par les discriminants des
    bases de $L$ sur $K$ contenues dans $R$.
\end{defn}

\noindent Remarquons que cette définition coïncide avec 3.1.5. lorsque $R$ est
un $A$-module libre, en effet une $K$-base de $L$ contenue dans $R$ est en
particulier une famille libre de $R$ sur $A$ de rang $[L:K]$ donc le
discriminant de cette base est associé dans $A$ à n'importe quelle base de $R$
sur $A$ comme on l'a vu précédemment. De plus par 3.1.3. et 3.1.4. l'idéal
discriminant est un idéal entier non nul de $A$.

On énonce maintenant le résultat principal de cette partie qui permet de
déterminer quels sont les idéaux de $A$ qui se ramifient dans $R$. On
justifiera ensuite la définition d'idéal discriminant et on donnera une preuve
de ce résultat.

\begin{thm}
    Un idéal premier $\mathfrak p$ de $A$ se ramifie dans $R$ si et seulement si
    $\mathfrak p\mid \mathfrak O_{R/A}$
\end{thm}

\begin{cor}
    L'ensemble des idéaux premiers $\mathfrak p$ de $A$ qui se ramifient dans $R$
    est fini.
\end{cor}

\noindent Le corollaire est une conséquence immédiate du fait que $\mathfrak
    O_{R/A}\ne(0)$ comme on l'a vu dans la dernière remarque.

\begin{defn}
    On dit qu'un anneau est réduit s'il n'a d'autre élément nilpotent que $0$.
\end{defn}

\noindent On remarque dans un premier temps que l'isomorphisme $R/R\mathfrak
    p\cong \prod_{i=1}^qR/\mathfrak P_i^{e_i}$ définit de manière évidente une
structure de $A/\mathfrak p$-algèbre de dimension finie sur $R/R\mathfrak p$.
La condition $\mathfrak p$ se ramifie dans $R$ se réécrit donc $R/R\mathfrak p$
est non réduite et on a le lemme suivant :

\begin{lem}(Lemme 1)
    Soit $F$ un corps fini ou de caractéristique 0 et $\mathfrak F$ une $F$-algèbre
    de dimension finie sur $F$. Pour que $\mathfrak F$ soit réduite, il faut et il
    suffit que $\mathfrak D_{\mathfrak F/F}\ne (0)$.
\end{lem}

\noindent Ici $\mathfrak D_{\mathfrak F/F}$ désigne le discriminant de
$\mathfrak F$ sur $F$ au sens de 3.1.5.. On aura besoin des lemmes suivants :

\begin{lem}
    Soit $B$ un anneau noethérien réduit. Alors l'idéal $(0)$ est intersection
    finie d'idéaux premiers.
\end{lem}

{\bf \noindent Preuve :} Par le lemme 3.2.3. $(0)$ est un produit d'idéaux
premiers $\mathfrak p_1^{e_1}...\mathfrak p_q^{e_q}$. Soit alors $x\in\mathfrak
    p_1\cap...\cap\mathfrak p_q$, on a $x^{e_1+\cdots+e_q}=0$ d'où $x=0$ car $B$
est réduit.\qed

\begin{lem}
    Soient $A$ un anneau, $B_1$,...,$B_q$ des anneaux contenant $A$ qui sont des
    $A$-modules libres de rang fini, et $B=\prod_{i=1}^qB_i$ leur anneau produit.
    Alors $\mathfrak D_{B/A}=\prod_{i=1}^q\mathfrak D_{B_i/A}$.
\end{lem}

{\bf \noindent Preuve :} On se ramène au cas $q=2$ par récurrence. Soient alors
$(x_1,...,x_m),(y_1,...,y_n)$ des $A$-bases de $B_1\times\{0\}$ et $\{0\}\times
    B_2$, $(x_1,...,x_m,y_1,...,y_n)$ est alors une base de $B_1\times B_2$ et
$Tr(x_iy_j)=Tr(0)=0$. D'où $D(x_1,...,x_m,y_1,...,y_n)$ s'écrit
$$\begin{vmatrix}
        Tr(x_ix_{i'}) & 0             \\
        0             & Tr(y_jy_{j'})
    \end{vmatrix}$$
ce qui donne le résultat voulu.
\\ \newline

{\bf \noindent Preuve du lemme 1 :} Supposons d'abord $\mathfrak F$ non réduite
et soit $x\in\mathfrak F$ un élément nilpotent non nul. On pose $x_1=x$ et on
complète ce début de base en une base $(x_1,...,x_n)$ de $\mathfrak F$ sur $F$.
Alors $x_1x_j$ est nilpotent, et ainsi la multiplication par $x_1x_j$ est un
endomorphisme nilpotent qui a donc pour seule valeur propre $0$, d'où
$Tr(x_1x_j)=0$ pour chaque $j$. La matrice $(Tr(x_ix_j))$ a donc une ligne
nulle, d'où $\mathfrak D_{\mathfrak F/F}=(D(x_1,...,x_n))=(0)$.\\ \indent
Réciproquement, supposons $\mathfrak F$ réduite. Alors
$(0)=\bigcap_{i=1}^q\mathfrak P_i$ où les $P_i$ sont des idéaux premiers de
$\mathfrak F$. Comme $\mathfrak F/\mathfrak P_i$ est une algèbre intègre de
dimension finie sur $F$, c'est un corps. Donc $\mathfrak P_i$ est un idéal
maximal de $\mathfrak F$, de sorte que $\mathfrak P_i~+~\mathfrak P_j=\mathfrak
    F$ pour $i\ne j$. Ainsi $\mathfrak F$ est isomorphe au produit
$\prod_{i=1}^q\mathfrak F/\mathfrak P_i$. On a donc $\mathfrak
    D_{L/K}=\prod_{i=1}^q\mathfrak D_{(\mathfrak F/\mathfrak P_i)/K)}$ par 3.3.8..
Or les $\mathfrak D_{(\mathfrak F/\mathfrak P_i)/K)}$ sont non nuls par 2.1.6.,
d'où $D_{\mathfrak F/F}\ne 0$.\qed

On peut donc réduire davantage la condition $\mathfrak p$ se ramifie dans $R$
par $\mathfrak D_{(R/\mathfrak pR)/(A/\mathfrak p)}=0$ car $A/\mathfrak p$ est
un corps fini. Or posons $S=A\backslash\mathfrak p$, $A'=S^{-1}A$, $R'=S^{-1}R$
et $\mathfrak p'=\mathfrak pA'$. Alors $A'$ est un anneau principal, $R'$ est
un $A'$-module libre, et on a $A/\mathfrak p\cong A'/\mathfrak p'$ et
$R/\mathfrak pR\cong R'/\mathfrak p'R'$. En notant alors $(e_1,...,e_n)$ une
base de $R'$ sur $A'$, la relation $\mathfrak D_{(R/\mathfrak pR)/(A/\mathfrak
            p)}=0$ équivaut à $D(e_1,...,e_n)\in \mathfrak p'$. En effet il est clair que
$D(\bar{e_1},...,\bar{e_n})=\overline{D(e_1,...,e_n)}$ où pour $x\in R'$,
$\bar{x}$ désigne l'image de $x$ dans le quotient $R'/\mathfrak p'R'$ et
$\overline{D(e_1,...,e_n)}$ désigne l'image dans le quotient $A'/\mathfrak
    p'A'$ de $D(e_1,...,e_n)$. Ceci étant, si $D(e_1,...,e_n)\in\mathfrak p'$ et si
$(x_1,...,x_n)$ est une base de $L$ sur $K$ contenue dans $R$, on a $x_i=\sum
    a'_{ij}e_j$ avec $a'_{ij}\in A'$, d'où
$D(x_1,...,x_n)=det(a_{ij})^2D(e_1,...,e_n)\in\mathfrak p'$. Comme $\mathfrak
    p'\cap A=\mathfrak p$, on en déduit $D(x_1,...,x_n)\in \mathfrak p$ et
$\mathfrak D_{R/A}\subset \mathfrak p$. Réciproquement, si $\mathfrak
    D_{R/A}\subset\mathfrak p$, on a $D(e_1,...,e_n)\subset \mathfrak p'$ car on
peut écrire $e_i=y_i/s$ avec $y_i\in R$ et $s\in S$ pour $1\leq i\leq n$. Ainsi
$$D(e_1,...,e_n)=s^{-2}D(x_1,...,x_n)\in A'\mathfrak D_{R/A}\subset \mathfrak
    pA'=\mathfrak p'$$ Ce qui conclut la preuve du théorème.\qed

\section{Le théorème de Chebotarev }
\indent Cette partie présente le théorème de Chebotarev et motive les
prochaines parties. On appelle discriminant d'un polynôme le discriminant usuel
obtenu à partir du résultant et on le note $\Delta(P)$. On a montré en 1.3 que
toute progression arithmétique de la forme $a+bm$, $pgcd(a,b)=1$, contient une
infinité de nombre premiers, c'est le théorème de la progression arithmétique
de Dirichlet. Le théorème de Chebotarev en est une généralisation. Il a été
d'abord conjecturé par Frobenius en tant que généralisation de son propre
théorème que l'on va énoncer ici sans preuve afin d'introduire la substitution
de Frobenius d'un nombre premier $p$.  \\  \indent Soit $P\in\Z[X]$ de
discriminant non nul. Le polynôme $P$ définit une extension galoisienne $K$ de
$\Q$ et son groupe de Galois $G$ peut être vu comme un sous-groupe de $S_n$ où
$n=deg(P)$. On sait de plus que tout élément de $S_n$ se décompose en produit
de cycles disjoint, par exemple la permutation $\sigma_1=(6)(7)(45)(123)\in
    S_7$ est décomposée en produit de cycles disjoints. D'un autre côté pour $p$ un
nombre premier tel que $p\nmid \Delta(P)$, $P$ mod $p$ se décompose en produit
de facteurs irréductibles dont les degrés forment une partition de $n$. On
introduit alors deux notations :

\begin{defn}
    On appelle type de $\sigma$, et on note $type(\sigma)$, la partition de $n$ à
    laquelle elle correspond où on définit une partition de $n$ comme un t-uple
    $$(i_1,...,i_k)\quad \textrm{  vérifiant}\quad i_j\leq i_l$$ pour tout $j\leq
        l$ et $$\sum_{j=1}^{k}i_j=n.$$ On a alors $type(\sigma_1)=(1,1,2,3)$.
\end{defn}

\begin{defn}
    On appelle type de $P$ par rapport à $p$, et on note $type(P,p)$, la partition
    formée par les degrés des facteurs irréductibles de $P$ $mod~p$.
\end{defn}
Avec ces notations, le théorème de Frobenius s'énonce ainsi.

\begin{thm}
    Soit $\sigma\in G$. La densité des nombre premiers $p$ tels que
    $type(P,p)=type(\sigma)$ existe et vaut $\frac{\mid T(\sigma)\mid}{\mid G\mid}$
    où $T(\sigma)=\{\epsilon\in G\mid type(\epsilon)=type(\sigma)\}$.
\end{thm}

Le théorème de Frobenius affirme que la "proportion" de nombres premiers $p$
tels que le type de $P$ par rapport $p$ est donnée est proportionnelle au
nombre d'éléments de G ayant cette même décomposition. \\ \indent On aimerait
maintenant savoir si cet énoncé implique celui de Dirichlet. Pour cela on peut
tenter de trouver un polynôme $P$ tel que les éléments du groupe de Galois
qu'il définit ne dépendent un à un que d'une classe $mod$ $p$. Prenons
$P=X^{12}-1$, on exclut $p=2,3$ car $2,3\mid\Delta(P)$. On considère maintenant
l'extension correspondant à $P$ de $F_p$ que l'on note $F_p(\alpha)$. Le type
de $P$ par rapport à $p$ est entièrement déterminé par l'action de $Frob_p$, le
morphisme de Frobenius, sur les racines de $P$. En effet on sait par la théorie
de Galois que le groupe de Galois d'un polynôme agit transitivement sur ses
facteurs irréductibles et que  $Gal(F_p(\alpha)/F_p)$ est engendré par
$Frob_p$. En particulier, l'orbite de l'action de $Frob_p$ correspond à
l'orbite de la multiplication par $p$ dans $\Z/12\Z$. D'où $type(P,p)$ ne
dépend que de $p$ $mod$ $12$ et par un calcul rapide on a les types suivants
$$\qquad\quad~~~ p\equiv 1~mod~12~:\qquad 1,1,1,1,1,1,1,1,1,1,1,1$$ $$~p\equiv
    5~mod~12~:\qquad1,1,1,1,2,2,2,2$$ $$\quad~p\equiv
    7~mod~12~:\qquad1,1,1,1,1,1,2,2,2$$ $$p\equiv
    11~mod~12~:\qquad1,1,2,2,2,2,2$$Alors le théorème de Frobenius pour $X^{12}-1$
implique le théorème de Dirichlet dans le cas $(a,12)$ tel que $pgcd(a,12)=1$.
Mais si l'on tente de généraliser ce résultat on tombe par exemple sur le cas
$X^{10}-1$ ou cette fois on obtient les types suivants
$$\qquad\quad\quad~~p\equiv 1~mod~10~:\qquad1,1,1,1,1,1,1,1,1,1$$ $$p\equiv
    3,7~mod~10~:\qquad 1,1,4,4$$ $$\quad p\equiv 1~mod~10~:\qquad1,1,2,2,2,2$$ Les
cas $p\equiv 3~mod~10$ et $p\equiv 7~mod~10$ ne sont plus distingués donc le
théorème de Frobenius pour $P=X^m-1$ n'implique par le théorème de Dirichlet.
On va maintenant formuler une généralisation du théorème de Frobenius qui
revient au théorème de Dirichlet dans le cas $X^m-1$.

\subsection{Groupe de décomposition et groupe d'inertie}
On utilise les mêmes notations que dans 3.1. Soit $A$ un anneau de Dedekind de
caractéristique $0$, $K$ son corps des fractions, $L$ une extension galoisienne
de degré $n$ de $K$, $G$ son groupe de Galois et $R$ la fermeture intégrale de
$A$ dans $L$. Dans la suite, pour $\mathfrak p$ un idéal premier non nul de $A$
et $\mathfrak P$ un idéal premier de $R$, on dira que $\mathfrak P\mid
    \mathfrak p$ si $\mathfrak P$ apparaît dans la décomposition de $R\mathfrak p$
dans $R$.

\begin{prop}
    Le groupe de Galois $G$ agit sur $I_e(R)$ l'ensemble des idéaux de $R$ par
    \begin{align*}
        G\times I_e(R) & \longrightarrow I_e(R)                        \\
        (\sigma,I)~    & \longmapsto~ \sigma(I)=\{\sigma(x) : x\in I\}
    \end{align*}
\end{prop}
{\bf \noindent Preuve :} Il suffit de remarquer que si $x\in R$ est entier sur
$A$, $\sigma(x)$ aussi. Pour le voir on prend $P\in A[X]$ unitaire tel que
$P(x)=0$. Alors $\sigma(P(x))=P(\sigma(x))=0$ d'où $\sigma(R)\subset R$.\qed \\

\noindent En fait on a $\sigma(R)=R$ pour tout $\sigma\in G$. Il suffit
d'appliquer le même raisonnement avec $\sigma^{-1}$, ce qui donne
$\sigma^{-1}(R)\subset R$ puis $R\subset\sigma(R)$. De plus, si $\mathfrak p$
est un idéal premier non nul de $A$ et $\mathfrak P\mid\mathfrak p$ alors
$\sigma(\mathfrak P)\mid \mathfrak p$. En effet on applique $\sigma$ à la
relation $\mathfrak P\cap A=\mathfrak p$ et le résultat est immédiat. On dira
que $\mathfrak P$ et $\sigma(\mathfrak P)$ sont conjugués. On veut maintenant
maintenant montrer que l'ensemble des $\mathfrak P$ tel que $\mathfrak P\mid
    \mathfrak p$ n'est autre que l'ensemble des conjugués de $\mathfrak P$ :

\begin{prop}
    Soit $\mathfrak p$ un idéal premier non nul de $A$. Le groupe de Galois $G$
    agit transitivement sur l'ensemble $\{\mathfrak P\mid \mathfrak p\}$.
\end{prop}
\noindent On aura besoin du lemme suivant :
\begin{lem}
    Soit $\mathfrak P_1,...,\mathfrak P_r$ des idéaux premiers de $R$ et $\mathfrak
        b$ un idéal de $R$ tel que $\mathfrak b\not\subset \mathfrak P_i$ pour tout
    $i$. Alors il existe $b\in\mathfrak b$ tel que $b\notin\mathfrak P_i$ pour tout
    $i$.
\end{lem}
{\bf \noindent Preuve du Lemme :} On peut supposer que les $\mathfrak P_i$ sont
tels que $\mathfrak P_i\not\subset\mathfrak P_j$ pour tout $i\ne j$, on peut
supposer de plus que les $\mathfrak P_i$ sont maximaux. Soit
$x_{ij}\in\mathfrak P_j$ tel  que $x_{ij}\notin\mathfrak P_i$. Comme $\mathfrak
    b\not\subset\mathfrak P_i$, il existe $a_i\in \mathfrak b$ tel que $a_i\notin
    \mathfrak P_i$ pour chaque $i$. On pose alors $$b_i=a_i\prod_{j\ne i}x_{ij}$$On
a $b_i\in \mathfrak b$, $b_i\in\mathfrak P_j$ pour $j\ne i$ et $b_i\notin
    \mathfrak P_i$ car sinon $a_i$ ou l'un des $x_{ij}$, $j\ne i$, serait dans
$\mathfrak P_i$. On pose alors $b=b_1+b_2+...+b_q$, on a en notant
$\pi_{\mathfrak P_i}$ la projection canonique de $R$ dans $R/\mathfrak P_i$
:$$\pi_{\mathfrak P_i}(b)=\pi_{\mathfrak P_i}(b_i)\ne0$$ d'où le résultat.\qed
\newline

{\bf \noindent Preuve de la proposition :} Soit $\mathfrak P$ tel que
$\mathfrak P\mid \mathfrak p$. supposons qu'il existe $\mathfrak P'$ tel que
$\mathfrak P'\mid \mathfrak p$ et $\mathfrak P'$ n'est pas conjugué de
$\mathfrak P$. On applique le lemme à $\sigma(\mathfrak P)$ et $\mathfrak P'$
qui sont maximaux et distincts pour tout $\sigma\in G$. On obtient un élément
$x\in\mathfrak P'$ tel que $x\notin\sigma(\mathfrak P')$ pour tout $\sigma\in
    G$. On considère alors $$N(x)=\prod_{\sigma\in G}\sigma(x)$$ clairement
$N(x)\in\mathfrak P'$ et par 2.1.5., $N(x)\in A$ donc $N(x)\in\mathfrak P'\cap
    A=\mathfrak p$. D'autre part $x\notin\sigma^{-1}(\mathfrak P)$, d'où
$\sigma(x)\notin \mathfrak P$ pour tout $\sigma\in G$. Comme $\mathfrak P$ est
premier, on en déduit que $N(x)\notin\mathfrak P$, ce qui contredit
$N(x)\in\mathfrak p$. \qed\newline

\begin{cor}
    Les idéaux premiers $\mathfrak P$ tels que $\mathfrak P\mid\mathfrak p$ ont
    même indice de ramification et même degré résiduel.
\end{cor}
{\bf\noindent Preuve :} Soit $\mathfrak P_1$, $\mathfrak P_2$ tels que
$\mathfrak P_1,\mathfrak P_2\in\{\mathfrak P\mid\mathfrak p\}$ et soit
$\sigma\in G$ tel que $\mathfrak P_2=\sigma(\mathfrak P_1)$. La suite
$$1\rightarrow\mathfrak P_1\rightarrow R\overset{\sigma}{\rightarrow}
    R\rightarrow R/\mathfrak P_2\rightarrow 1$$ est exacte d'où $\mathfrak P_1$ et
$\mathfrak P_2$ ont même degré résiduel. En plus $R/R\mathfrak p\cong
    \prod_{\mathfrak P\mid\mathfrak p}R/\mathfrak P^{n_{\mathfrak p}(\mathfrak P)}$
par 3.2.5. et si $\pi$ désigne la projection de $R$ sur $R/R\mathfrak p $,
alors $\pi\circ\sigma$ permute les coordonnées des éléments du produit.
Combiner ce dernier résultat avec l'identité $\sigma(\mathfrak
    P_1^{n_{\mathfrak p}(\mathfrak P_1)})=\sigma(\mathfrak P_1)^{n_{\mathfrak
            p}(\mathfrak P_1)}=\mathfrak P_2^{n_{\mathfrak p}(\mathfrak P_1)}$ permet alors
de conclure sur les indices de ramification.\qed \newline

\noindent On notera $e$ l'indice de ramification des $\mathfrak P$ tels que
$\mathfrak P\mid\mathfrak p$ et $f$ leurs degrés résiduels. Alors en notant
$q=\lvert\{\mathfrak P\mid\mathfrak p\}\rvert$, on a $n=efq$.

\begin{defn}
    Soit $\mathfrak P_1\in\{\mathfrak P\mid \mathfrak p\}$. On appelle groupe de
    décomposition de $\mathfrak P_1$ et on note $D(\mathfrak P_1)$ le sous groupe
    de $G$ formé des $\sigma$ tels que $\sigma(\mathfrak P_1)=\mathfrak P_1$.
\end{defn}

\noindent Pour $\sigma\in D(\mathfrak P)$ on peut définir un $A/\mathfrak
    p$-automorphisme de $R/\mathfrak P$ que l'on note $\overline{\sigma}$ par
$\overline{\sigma}(\overline{x}):=\overline{\sigma(x)}$, par définition de
$D(\mathfrak P)$ on vérifie facilement que $\overline{\sigma}$ définit bien un
$A/\mathfrak p$-automorphisme de $R/\mathfrak P$. On considère maintenant le
morphisme de groupes $$\sigma\longmapsto \overline{\sigma}$$il a pour noyau
l'ensemble des $\sigma\in D(\mathfrak P)$ tels que $\sigma(x)-x\in\mathfrak P$
pour tout $x\in R$.

\begin{defn}
    On appelle groupe d'inertie de $\mathfrak P$ et on note $I(\mathfrak P)$ le
    groupe $$I(\mathfrak P)=\{\sigma\in D(\mathfrak P)\mid \sigma(x)-x\in\mathfrak
        P~\forall x\in\R\}$$
\end{defn}
\noindent C'est aussi le noyau du morphisme de groupe défini précédemment et
donc un sous-groupe distingué de $D(\mathfrak P)$.

\noindent On admet maintenant le résultat suivant qui est traité en [2] (chap
6, p.106).

\begin{prop}
    L'extension $R/\mathfrak P$ de $A/\mathfrak p$ est galoisienne de degré $f$, de
    groupe de Galois $D/I$.
\end{prop}
\begin{cor}
    Pour que $\mathfrak p\in\mathfrak A(A)$ ne se ramifie pas dans $R$, il faut et
    il suffit que le groupe d'inertie $I(\mathfrak P)$ soit réduit à l'identité.
\end{cor}
\noindent En effet, par la proposition 4.1.7. le groupe d'inertie à pour
cardinal exactement l'indice de ramification de $\mathfrak p$. On peut
maintenant introduire la substitution de Frobenius d'un idéal premier
$\mathfrak p$.

\subsection{La substitution de Frobenius}
On garde les notations du chapitre précédent. Soit $\mathfrak p$ un idéal
premier de $A$ qui ne se ramifie pas dans $R$ et $\mathfrak P$ un idéal premier
de $R$ tel que $\mathfrak P\mid \mathfrak p$. Par 4.1.8. le groupe d'inertie de
$\mathfrak P$ est trivial et donc le groupe de décomposition de $\mathfrak P$
est canoniquement isomorphe au groupe de Galois $G'$ de $R/\mathfrak
    P/A/\mathfrak p$. En notant $\bar{\pi}$ cet isomorphisme on définit maintenant
la substitution de Frobenius.

\begin{defn}
    On appelle substitution de Frobenius de $\mathfrak p$ le générateur
    $\sigma_{\mathfrak P}$ de $D(\mathfrak P)$ pour chaque $\mathfrak P\mid
        \mathfrak p$, tel que $\bar{\pi}(\sigma)=(\bar{x}\mapsto \bar{x}^{N(\mathfrak
            p)})$ est le générateur privilégié de $G'$. On la note alors $(\mathfrak P,
        L/K)$.
\end{defn}

\begin{prop}
    Soit $\sigma(\mathfrak P_1)=\mathfrak P$ pour $\sigma\in G$. Alors
    $D(\mathfrak P)=\sigma D(\mathfrak P_1)\sigma^{-1}$.
\end{prop}
\noindent La substitution de Frobenius est alors définie à classe de
conjugaison près et en particulier dans le cas où $G$ est abélien, celle-ci ne
dépend que de $\mathfrak p$. On la note alors $(\mathfrak p, L/K)$. On
considère maintenant $K\subset E\subset L$ une extension galoisienne
intermédiaire de $L/K$, on a

\begin{prop}
    On note $f$ le degré résiduel de $\mathfrak P\cap E$ sur $K$. Alors
    $$(\mathfrak P, L/E)=(\mathfrak P,L/K)^f$$ et $$(\mathfrak P\cap E,
        E/K)=(\mathfrak P, L/K)\rvert_E$$
\end{prop}
{\bf \noindent Preuve :} On pourra se réferer à [2] (chap. 6, p.108) pour une
preuve.\qed
\newline

\noindent On étudie maintenant le cas où $L/K$ est cyclotomique.

\begin{prop}
    Lorsque $L=K(\zeta)/K$ est cyclotomique avec $\zeta$ une racine $m$-ème de
    l'unité, $(\mathfrak p, L/K)$ ne dépend que de la norme de $\mathfrak p$ modulo
    $m$.
\end{prop}
{\bf \noindent Preuve :} Tout morphisme d'une extension cyclotomique est
entièrement défini par son action sur $\zeta$. Ici $(\mathfrak p, L/K)$ envoie
$\zeta$ sur $\zeta^{N(\mathfrak p)}$ et donc ne dépend que de l'orbite de
$N(\mathfrak p)$ modulo $m$ d'où le résultat.\qed \newline

\noindent Il est maintenant nécessaire de lier les notations $\Delta(P)$ et
$\mathfrak O(L/K)$. Lorsque $\Delta(P)\ne 0$ et $L$ est le corps de
décomposition de $P$ on a la proposition :
\begin{prop}
    $$\mathfrak O(L/K)=(\Delta(P)).$$
\end{prop}

\noindent On peut maintenant énoncer le théorème de Chebotarev.
\begin{thm}
    (théorème de densité de Chebotarev) Soit $P\in F[X]$ unitaire de discriminant
    non nul, $K$ un corps de décomposition de $P$ et $\sigma\in G=Gal(K/F)$. La
    densité de l'ensemble des idéaux premiers $\mathfrak p$ tels que
    $\Delta(P)\not\subset\mathfrak p$ et tels qu'il existe $\mathfrak P\mid
        \mathfrak p$ tel que $\sigma_{\mathfrak P}\in C(\sigma)$, où $C(\sigma)$ est la
    classe de conjugaison de $\sigma$, existe et vaut $\frac{\mid
            C(\sigma)\mid}{\mid G\mid}$.
\end{thm}
\noindent Remarquons Les idéaux premiers $\mathfrak p$ étant en nombre fini.
Ceux-ci n'influent pas sur la densité voulue. La preuve du théorème nécessite
de se placer dans le cas ou le corps de base n'est plus $\Q$ mais un corps de
nombre $F$. La notion de densité utilisée en première partie n'est donc plus
valable, on va donc considérer une nouvelle fonction zêta, la fonction zêta de
Dedekind.

\subsection{Fonctions L de Dirichlet généralisées}
Soit $A$ l'anneau des entiers d'un corps de nombre $K$. On suppose ici connus
les résultats usuels sur les valeurs absolues de corps de nombres. On note
$M_K$ l'ensemble des valeurs absolues normalisées de $K$ i.e. telles que leur
restriction à $\Q$ est une valeur absolue réelle usuelle ou une valeur absolue
$p$-adique.

\begin{defn}
    On appelle $cycle$ de $K$ tout produit formel $$\mathfrak m=\prod_{M_K}
        v^{m(v)}$$ où les $m(v)$ sont positifs et presque tous nuls.
\end{defn}
\noindent Si $m(v)>0$ pour $v$ réelle on se restreint au cas où $m(v)=1$. On
utilisera la notation $\mathfrak m=\prod_{M_K\backslash
        \{v_{\infty}\}}\mathfrak p^{m(\mathfrak p)}\times\mathfrak p_{\infty}$ où
$\mathfrak p_{\infty}$ désigne la valeur absolue réelle. Lorsque
$m(p_{\infty})=0$ on utilisera sans distinction la notation $\mathfrak m$ pour
désigner l'idéal de $K$ correspondant. L'étude de la fonction zêta de Dedekind
est un cas particulier de l'étude des fonctions $L$ de Dirichlet généralisées à
un corps de nombres que l'on va parcourir ici. Soit $\mathfrak
    m=\prod_{i=1}^q\mathfrak p_i^{e_i}$ un idéal de $A$. On appelle $I_{\mathfrak
            m}(A)$ (resp. $P_{\mathfrak m}(A)$) le groupe des idéaux (resp. idéaux
principaux) premiers à $\mathfrak m$. On rappelle que l'on a le théorème :

\begin{thm}
    Le groupe quotient $\dfrac{I(A)}{P(A)}$ est fini.
\end{thm}

\begin{cor}

    On a $$I_{\mathfrak m}(A)/P_{\mathfrak m}(A)\cong I(A)/P(A)$$ et chaque classe
    de $I_{\mathfrak m}(A)/P_{\mathfrak m}(A)$ admet un représentant entier.
\end{cor}
{\bf \noindent Preuve :} Il suffit de montrer que chaque classe de $I(A)/P(A)$
à un représentant dans $I_{\mathfrak m}(A)$. Soit donc $\mathfrak K$ une classe
de $I(A)/P(A)$ et $\mathfrak a\in \mathfrak K$, on peut le supposer entier. On
commence, à l'aide du lemme chinois, par résoudre les congruences $$
    \alpha\equiv p_i^{e_i}~mod~\mathfrak p^{e_i+1}$$ où les $p_i\in\mathfrak p_i$
est premier pour chaque $i$. Alors $\mathfrak a(\alpha^{-1})\in I_{\mathfrak
            m}(A)$ ce qui conclut la preuve. \qed \\

\noindent On reprend maintenant la notation $\mathfrak m$ pour désigner
$\mathfrak m\times v_{\infty}$. On note maintenant $P_{\mathfrak
            m}:=P_{\mathfrak m}(A)$ lorsque aucune confusion n'est à craindre, de même pour
$I_{\mathfrak m}(A)$. On considère maintenant le sous-groupe $P(\mathfrak m)$
de $P_{\mathfrak m}$ dont les éléments sont ceux de $P_{\mathfrak m}$ tels que
pour toute valeur absolue réelle $v$ induite par $\sigma_v\in G$ telle que
$m(v)>0$ dans $\mathfrak m$. On a $\sigma_v(P(\mathfrak m))\subset \R^+$. On
considère maintenant le quotient $I_{\mathfrak m}/P(\mathfrak m)$.
\begin{thm}
    Le groupe $I_{\mathfrak m}/P(\mathfrak m)$ est fini.
\end{thm}
{\bf \noindent Preuve :} La preuve peut être trouvée dans [3](chap. VI,
p.127).\qed

Il s'avère que ce groupe généralise la notion de progression arithmétique
$mod~m$ à un corps de nombre quelconque et justifie l'appellation fonction $L$
de Dirichlet généralisées.

On considère $\chi$ un caractère du groupe abélien fini $I_{\mathfrak
            m}/P(\mathfrak m)$
\begin{defn}
    Soit $I$ un idéal entier non nul de $A$. On appelle norme de $I$ et on note
    $N(I)$ le nombre $card(A/I)$.
\end{defn}

\begin{prop}
    La norme d'idéal est multiplicative.
\end{prop}

\begin{prop}
    Soit $x\in A$, alors $\lvert N(x)\rvert=N(Ax)$, où $N(x)$ désigne la norme de
    $x$ au sens de 2.1.4.
\end{prop}

\noindent On étend $\chi$ en un caractère de $I_{e}(A)$ de la même manière
qu'en partie $1$.

\begin{defn}
    On appelle fonction $L$ de $K$ associée à $\chi$ la série
    $$L(\chi,s)=\sum_{\mathfrak a}\dfrac{\chi(\mathfrak a)}{N(\mathfrak a)^s}$$ où
    la somme parcourt les idéaux non nuls de $A$.
\end{defn}
\begin{defn}
    On appelle fonction zêta de Dedekind de $K$ et on note $\zeta_K$ la série
    $L(1,s)$.
\end{defn}

\noindent On cherche à déterminer $N(\mathfrak p)$ pour $\mathfrak p$ un idéal
premier de $K$. On sait que pour $\mathfrak p$ un idéal premier de $A$,
$\mathfrak p\cap \Z=p\Z$ pour un certain $p$ premier. On a vu en 3.2 que $A/Ap$
est un $\Z/p\Z$-espace vectoriel de dimension finie égale à $[K:\Q]$. En
particulier $A/Ap$ à pour cardinal $p^{[K:\Q]}$. Enfin
$$N(Ap)=\prod_{i=1}^qN(\mathfrak p_i)^{e_i}$$ ou les $\mathfrak p_i$ sont par
3.2.1. exactement les idéaux premiers tels que $\mathfrak p\cap \Z=p\Z$ d'où
$N(\mathfrak p_i)=p^f_i$ avec $1\leq f_i\leq [K:\Q]$, $f_i$ est en fait par
3.2.4. le degré résiduel de $\mathfrak p_i$ sur $\Z$. On peut maintenant
prouver la proposition suivante :

\begin{prop}
    Pour tout caractère $\chi$ de $I_{\mathfrak m}/P(\mathfrak m)$. La fonction $L$
    associée à $\chi$ converge absolument sur $D(1)$, y est holomorphe et on a
    $$L(\chi, s)=\prod_{\mathfrak p}\dfrac{1}{1-\frac{\chi(\mathfrak
                a)}{N(\mathfrak p)}}$$ où le produit parcourt les idéaux premiers non nuls de
    $A$.
\end{prop}

{\bf \noindent Preuve :} On note log la branche principale du logarithme. On
prend formellement le logarithme du produit $$f(s)=\prod_{\mathfrak
        p}\dfrac{1}{1-\frac{\chi(\mathfrak a)}{N(\mathfrak p)}}$$ ce qui donne la série

$$\textrm{log}(f(s))=\sum_{\mathfrak
        p}\sum_{k=1}^{\infty}\dfrac{\chi(\mathfrak p)}{kN(\mathfrak p)^{-ks}}$$

\noindent et on a

$$\textrm{log}(f(s))=\sum_{\mathfrak
        p}\sum_{k=1}^{\infty}\dfrac{\chi(\mathfrak p)}{kN(\mathfrak
        p)^{-ks}}=\sum_{p}\sum_{\mathfrak p\in
        E_p}\sum_{k=1}^{\infty}\dfrac{\chi(\mathfrak p)}{kN(\mathfrak p)^{-ks}}$$

\noindent d'où

$$\lvert \textrm{log}(f(s))\rvert\leq
    \sum_{k,p}\dfrac{1}{kp^{k[K:\Q]Re(s)}}\leq [K:\Q]\times\textrm{log}(\zeta(s))$$

\noindent dans $D(1)$ ce qui donne la convergence absolue et uniforme dans
$D(1)$ du produit $$\prod_{\mathfrak p}\dfrac{1}{1-\frac{\chi(\mathfrak
            p)}{N(\mathfrak p)}}=\textrm{exp}\Big(\sum_{\mathfrak
            p}\sum_{k=1}^{\infty}\dfrac{\chi(\mathfrak p)}{kN(\mathfrak
            p)^{-ks}}\Big).$$Par l'unique décomposition en idéaux premiers des idéaux de
$A$, on montre ensuite l'égalité exactement de la même manière que pour le
produit eulérien des fonctions $L$ de Dirichlet vue en 1.1.7.. \qed \\

\noindent On voudrait maintenant montrer que $L(1,s)$ admet un pôle simple en
$s=1$.

\noindent On sait que chaque classe admet un représentant entier sur $A$. On
commence donc par décomposer $L(\chi,s)$ en une somme finie
$$L(\chi,s)=\sum_{\mathfrak K}L_{\mathfrak K}(\chi,s)$$ où la somme parcourt
les classes $\mathfrak K$ de ${I_{\mathfrak m}}/{P(\mathfrak m)}$ et
$\zeta_{\mathfrak K,K}(s)=\sum_{\mathfrak a\in\mathfrak K\cap
        I_e(A)}\dfrac{1}{N(\mathfrak a)^s}$. On peut en fait se ramener à l'étude de la
fonction zêta de Riemann.
\begin{thm}
    Soit $\mathfrak K$ une classe de ${I_{\mathfrak m}}/{P(\mathfrak m)}$. Si l'on
    note $j(\mathfrak K,t)$ le nombre d'idéaux $\mathfrak a$ de $A$ tels que
    $N(\mathfrak a)\leq t$ alors on a $$j(\mathfrak
        K,t)=Ct+O(t^{1-(1/[K:\Q])}),~t\rightarrow \infty,$$ où $C>0$ est une constante
    qui ne dépend que de $K$.
\end{thm}

\begin{cor}
    La fonction zêta de $K$ a un pôle simple en $s=1$.
\end{cor}
\noindent La preuve est admise et traitée dans [3](chap. VIII p.161).

\begin{cor}
    On a $$\sum_{\mathfrak p\in P_K}1/N(\mathfrak
        p)^s\sim_1\textrm{log}(\frac{1}{s-1})$$
\end{cor}
\noindent La preuve est la même que dans le cas $K=\Q$.\qed\\

\begin{defn}
    On définit alors la densité analytique sur le corps $K$ d'un ensemble $ P_1$
    d'idéaux premiers de $K$ par la limite, si elle existe,
    $$d(P_1)=\lim_{s\rightarrow 1, s>1}\big(\sum_{\mathfrak p\in
                P_1}\frac{1}{N(\mathfrak p)^s}\big)/\textrm{log}(\frac{1}{s-1})$$
\end{defn}

\noindent On admet maintenant deux théorèmes sans preuves qui sont conséquences
d'un théorème plus profond de la théorie du corps de classes que l'on énonce
maintenant :
\begin{thm}(Loi de réciprocité d'Artin)
    On suppose que $L/K$ est abélienne. Il existe $H$ un sous-groupe du quotient
    ${I_{\mathfrak m}}/{P(\mathfrak m)}$ tel que ${I_{\mathfrak m}}/H\cong
        Gal(L/K)$ où l'isomorphisme est donné par $\mathfrak p\longmapsto (\mathfrak p,
        L/K)$ en étendant ce morphisme aux idéaux fractionnaires par multiplicativité.
\end{thm}
\begin{thm}
    Pour tout caractère $\chi\ne 1$ de ${I_{\mathfrak m}}/{P(\mathfrak m)}$, on a
    $L(\chi,1)\ne 0$.
\end{thm}

\noindent On a alors le théorème de Dirichlet généralisé :
\begin{cor}
    On note $a_{\mathfrak m}=[I_{\mathfrak m}:P(\mathfrak m)]$ l'ordre du groupe
    ${I_{\mathfrak m}}/{P(\mathfrak m)}$ et $\mathfrak K_0$ une de ses classes.
    Alors $d(\mathfrak K_0\cap\mathfrak A(A))=1/a_{\mathfrak m}$.
\end{cor}
{\bf \noindent Preuve :} On procède exactement de la même manière que pour la
preuve du théorème de Dirichlet en multipliant par $\chi(\mathfrak K_0^{-1})$
les relations à la place de $\chi(a^{-1})$, les relations d'orthogonalité
permettent alors de conclure. \qed \\ \newline En fait pour $P(\mathfrak
    m)\subset H\subset I_{\mathfrak m}$ un sous-groupe de $I_{\mathfrak m}$ le
résultat reste vrai en remplaçant $a_{\mathfrak m}$ par $h_{\mathfrak
            m}=[I_{\mathfrak m}:H]$, en particulier pour $H$ tel que ${I_{\mathfrak
                    m}}/H\cong Gal(L/K)$ ce qui donne la preuve du théorème de Chebotarev dans le
cas abélien. Cependant, la preuve originale de Chebotarev n'utilise pas de
théorie du corps de classes, on expose alors une partie de sa stratégie dans la
prochaine partie.

\subsection{Preuve du théorème}
Soit $K$ un corps de nombres et $L$ une extension galoisienne finie de $K$. La
preuve originale de Chebotarev est en trois partie. D'abord une réduction au
cas où $L/K$ est abélienne, puis le cas particulier où $L/K$ est cyclotomique
et enfin une résolution dans le cas abélien. On expose ici la première et la
troisième partie du raisonnement. On pose $n=[L:K]$ et $G=Gal(L/K)$.\\ \newline

{\bf \noindent Preuve :} Soit $\sigma\in G$, on note $E$ le sous corps de $L$
fixé par $<\sigma>$ et $f$ l'ordre de $\sigma$. On veut montrer que le théorème
est vrai pour $(L/K,C(\sigma))$ si et seulement si il est vrai pour $(L/E,
    \{\sigma\})$. Soit $P_{L/K,\mathfrak m}(\sigma)$ l'ensemble des idéaux premiers
$\mathfrak p$ de $K$ vérifiant l'hypothèse et premier à $\mathfrak m$. Soit de
plus $\bar{P}_{L/K,\mathfrak m}(\sigma)$ l'ensemble des idéaux premiers de $L$
tels que $\mathfrak P\mid \mathfrak p$ et $(\mathfrak P, L/K)=\sigma$ pour
$\mathfrak p\in P_{L/K,\mathfrak m}(\sigma)$. Soit maintenant $P_{L/E,\mathfrak
            m}'(\sigma)$ l'ensemble des idéaux premiers $\mathfrak q$ de $P_{L/E,\mathfrak
            m}(\sigma)$ ayant comme degré résiduel $1$ sur $\mathfrak q\cap K$. Alors 
            $\bar P_{L/K,\mathfrak m}(\sigma)$ et $P_{L/E,\mathfrak m}'(\sigma)$ sont en
bijection. En effet par 4.2.3., $(\mathfrak P\cap E,E/K)=(\mathfrak
    P,L/K)\rvert_E=\sigma\rvert_E$, mais par définition de $E$,
$\sigma\rvert_E=id_E$. Donc $D(\mathfrak q)$ est réduit à l'identité, d'où le
degré résiduel de $\mathfrak q$ sur $L$ est $1$. Si $N$ désigne le cardinal de
$\{\mathfrak P\mid \mathfrak p\}$, on a alors d'un côté
$$[L:K]=[L:E]\times[E:K]=f[E:K]$$ et de l'autre $$[L:K]=efN=fN$$ d'où $N=[L:K]$
et l'assertion voulue. Mais $P_{L/E,\mathfrak m}'(\sigma)$ ne diffère de
$P_{L/E,\mathfrak m}(\sigma)$ que par les idéaux premiers $\mathfrak q\mid
    \mathfrak p$ étant ramifiés ou ayant degré résiduel $>1$ sur $\Q$, mais ceux-ci
ont densité de Dirichlet $0$. Si on considère maintenant

\[\rho~:~ P_{L/E,\mathfrak m}'(\sigma) \longmapsto P_{L/K,\mathfrak
            m}'(\sigma)~;~
    \mathfrak q \longmapsto \mathfrak q\cap K\]

\noindent Alors $\rho$ est surjective et pour chaque $\mathfrak p\in
    P_{L/K,\mathfrak m}(\sigma) $, $\rho^{-1}(\mathfrak p)\cong \{\mathfrak P\in
    \bar P_{L/K,\mathfrak m}(\sigma)~;~ \mathfrak P\mid\mathfrak p\} $ par la
bijection précédente. Mais si l'on note $Z(\sigma)=\{\tau\in
    G~;~\tau\sigma=\sigma\tau\}$, alors $\{\mathfrak P\in \bar P_{L/K,\mathfrak
            m}(\sigma)~;~ \mathfrak P\mid\mathfrak p\}$ a même cardinal que
$Z(\sigma)/<\sigma>$. En effet les groupes de décomposition de $\mathfrak
    P_1,\mathfrak P_2\in\{\mathfrak P\in \bar P_{L/K,\mathfrak m}(\sigma)~;~
    \mathfrak P\mid\mathfrak p\}$ sont conjugués, disons $D(\mathfrak P_1)=\tau
    D(\mathfrak P_2)\tau^{-1}$, donc $(\mathfrak P_1, L/K)=(\mathfrak P_2,L/K)$ si
et seulement si $\tau$ commute avec $(\mathfrak P_2,L/K)$. Enfin on obtient
alors $$d(P_{L/K,\mathfrak
            m}(\sigma))=\dfrac{1}{[Z(\sigma):<\sigma>]}d(P_{L/E,\mathfrak m}'(\sigma))$$En
supposant que le théorème de Chebotarev est vrai pour $L/E$ et $\sigma$ on
obtient alors $$d(P_{L/K,\mathfrak m}(\sigma))=\dfrac{f}{\lvert
        Z(\sigma)\rvert}\frac{1}{f}=\dfrac{\lvert C(\sigma)\rvert}{\lvert G\rvert}$$
On peut donc se restreindre au cas ou $L/K$ est abélienne. Remarquons qu'avec
la loi de réciprocité d'Artin, on a prouvé le théorème de densité de
Chebotarev.

\noindent La seconde partie de la méthode de Chebotarev consiste à résoudre le
cas ou $L=K(\zeta)$, pour $\zeta$ une racine $m$-ème de l'unité tel que $(m)$
est premier à $ \mathfrak O_{L/K}$. La preuve suit une démarche similaire à
celle de Dirichlet qui est en fait le cas particulier $L=\Q(\zeta)$ et $K=\Q$.
Cependant elle nécessite les fonctions $L$ d'Artin qui généralisent les
fonctions $L$ abéliennes. Leur traitement étant assez technique on admet cette
partie qui ne contient pas les idées clés de Chebotarev.
\newline \newline
\indent Par la première partie on peut supposer $L/K$ abélienne. Soit $m$ un
nombre premier de $\N$ tel que $(m)$ est premier à $\mathfrak O_{L/K}$ et
$\zeta$ une racine primitive $m$-ème de l'unité. Alors $H=Gal(K(\zeta)/K)\cong
    (\Z/m\Z)^{\times}$ et $Gal(L(\zeta)/K)\cong G\times H$. Il est clair que si
$\mathfrak p$ un idéal premier de $K$ a pour substitution $(\sigma,\tau)$ dans
$G\times H$ alors il a pour substitution $\sigma$ dans $G$. Alors en notant
$d_{inf}$ la quantité définie de la même manière que la densité en remplaçant
la limite par une limite inférieure on a $$d_{inf}(P_{L/K,\mathfrak
            m}(\sigma))\geq\sum_{\tau\in H}d_{inf}(P_{L(\zeta)/K}( (\sigma,\tau)))$$

\noindent Soit maintenant $\sigma\in G$ et $\tau\in H$. On suppose que $n$
divise l'ordre de $\tau$, alors $<(\sigma, \tau)>$ et $G\times\{1\}$ ont
intersection triviale. En effet $(\sigma^k,\tau^k)=(\sigma',1)$ implique $n\mid
    k$ d'où $\sigma^k=1$. Alors le sous corps $E$ de $K(\zeta)$ fixé par
$<(\sigma,\tau)>$ vérifie $E(\zeta)=L(\zeta)$ de sorte que $E(\zeta)/L$ est
cyclotomique. En effet l'extension $L(\zeta)/E.K(\zeta)$ a pour groupe de
Galois $Gal(L(\zeta)/E)\cap Gal(L(\zeta)/K(\zeta))=\{id\}$. Par la deuxième
partie, la densité $d(P_{L(\zeta)/E}( (\sigma,\tau))$ existe et à la valeur
annoncée. Par la première partie c'est donc aussi le cas pour
$d(P_{L(\zeta)/K}( (\sigma,\tau)))$ qui vaut alors $1/\lvert G\rvert\lvert
    H\rvert$. En notant $H_n$ les éléments de $H$ d'ordre divisible par $n$ et en
sommant sur $\tau$ de la même manière que dans l'inégalité précédente on
obtient $$d_{inf}(P_{L/K,\mathfrak m}(\sigma))\geq \lvert H_n\rvert/\lvert
    G\rvert\lvert H\rvert$$ mais par le théorème de Dirichlet on peut toujours
trouver $m$ tel que $(m)$ est premier à $\mathfrak O_{L/K}$ et
$m\equiv1~mod~n^k$ pour tout $k$ d'où en faisant varier $m$ sur ceux-ci on
obtient $$d_{inf}(P_{L/K,\mathfrak m}(\sigma))\geq 1/\lvert G\rvert$$ On
définit maintenant $d_{sup}$ comme la densité de Dirichlet où l'on remplace la
limite par une limite supérieure. Le résultat étant vrai pour chaque $\sigma$
et les $P_{L/K,\mathfrak m}(\sigma)$ étant disjoint on a $1\geq\sum_{\sigma\in
        G}d_{inf}(P_{L/K,\mathfrak m}(\sigma))\geq\lvert G\rvert\times 1/\lvert
    G\rvert$ d'où $d_{inf}(P_{L/K,\mathfrak m}(\sigma))=1/\lvert G\rvert$. Alors
par définition de la densité et $$\sum_{\mathfrak p\in P_{L/K,\mathfrak
                m}(\sigma)}\frac{1}{N(\mathfrak p)^s}/\textrm{log}(\frac{1}{s-1})$$ étant
strictement décroissante sur $]1,+\infty[$, il suffit que la limite supérieure
existe pour qu'elle coïncide avec la limite inférieure. Ce qui est clair, d'où
$d(P_{L/K,\mathfrak m}(\sigma))$ existe et à la valeur annoncée.\qed \\

\newpage

{\bf Références :}\\
$[1]$ Jean-Pierre Serre, Cours d'arithmétique, Presses Universitaires de
France, 1994.\\
$[2]$ Pierre Samuel, Théorie algébrique des nombres, Hermann, 1997.\\
$[3]$ Serge Lang, Algebraic number theory, Springer-Verlag, New York, 1994.\\
$[4]$ Jürgen Neukirch, Algebraic number theory, Springer-Verlag, Berlin,
1999.\\
$[5]$ P. Stevenhagen et H.W. Lenstra, Jr., \textit{Chebotarëv and his Density
    Theorem}, \\https://www.math.leidenuniv.nl/~hwl/PUBLICATIONS/1996d/art.pdf

\end{document}
